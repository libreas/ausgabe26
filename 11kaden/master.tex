\documentclass[a4paper,
fontsize=11pt,
%headings=small,
oneside,
numbers=noperiodatend,
parskip=half-,
bibliography=totoc,
final
]{scrartcl}

\usepackage{synttree}
\usepackage{graphicx}
\setkeys{Gin}{width=.6\textwidth} %default pics size

\graphicspath{{./plots/}}
\usepackage[ngerman]{babel}
\usepackage[T1]{fontenc}
%\usepackage{amsmath}
\usepackage[utf8x]{inputenc}
\usepackage [hyphens]{url}
\usepackage{booktabs} 
\usepackage[left=2.4cm,right=2.4cm,top=2.3cm,bottom=2cm,includeheadfoot]{geometry}
\usepackage{eurosym}
\usepackage{multirow}
\usepackage[ngerman]{varioref}
\setcapindent{1em}
\renewcommand{\labelitemi}{--}
\usepackage{paralist}
\usepackage{pdfpages}
\usepackage{lscape}
\usepackage{float}
\usepackage{acronym}
\usepackage{eurosym}
\usepackage[babel]{csquotes}
\usepackage{longtable,lscape}
\usepackage{mathpazo}
\usepackage[flushmargin,ragged]{footmisc} % left align footnote

\usepackage{listings}

\urlstyle{same}  % don't use monospace font for urls

\usepackage[fleqn]{amsmath}

%adjust fontsize for part

\usepackage{sectsty}
\partfont{\large}

%Das BibTeX-Zeichen mit \BibTeX setzen:
\def\symbol#1{\char #1\relax}
\def\bsl{{\tt\symbol{'134}}}
\def\BibTeX{{\rm B\kern-.05em{\sc i\kern-.025em b}\kern-.08em
    T\kern-.1667em\lower.7ex\hbox{E}\kern-.125emX}}

\usepackage{fancyhdr}
\fancyhf{}
\pagestyle{fancyplain}
\fancyhead[R]{\thepage}

%meta
%meta

\fancyhead[L]{B. Kaden \\ %author
LIBREAS. Library Ideas, 26 (2014). % journal, issue, volume.
\href{http://nbn-resolving.de/urn:nbn:de:kobv:11-100222668
}{urn:nbn:de:kobv:11-100222668}} % urn
\fancyhead[R]{\thepage} %page number
\fancyfoot[L] {\textit{Creative Commons BY 3.0}} %licence
\fancyfoot[R] {\textit{ISSN: 1860-7950}}

\title{\LARGE{Konzepte für den Gegenwartsdiskurs. Heute: Linguistic Capitalism}} %title %title
\author{Ben Kaden} %author

\setcounter{page}{}

\usepackage[colorlinks, linkcolor=black,citecolor=black, urlcolor=blue,
breaklinks= true]{hyperref}

\date{}
\begin{document}

\maketitle
\thispagestyle{fancyplain} 

%abstracts

%body
Die natürliche Sprache wird sich in einer Sprachwelt, die von
Interaktionen mit sprachverarbeitenden Systemen wie den Diensten von
Google geprägt ist, verändern. Diese Idee eines durch digitales
Kommunizieren eintretenden Wandels im Sprachgebrauch allein dürfte nicht
sonderlich überraschen. Es überrascht vielleicht eher, wie wenig bewusst
wir in der Regel damit und den Bedingungen dieses möglichen Wandels
umgehen. Andererseits gibt es selbstverständlich Akteure, die in diesem
Zusammenhang sehr elaboriert agieren und wenig überraschend unter diesen
auch genau diejenigen, die unser Texthandeln im Digitalen als Basis
ihrer Geschäftsidee verwenden.

Der Digital Humanist Frederic Kaplan reflektiert in einem aktuellen
Aufsatz (Kaplan, 2014) darüber und über die möglichen Auswirkungen
algorithmenbasierter Sprachvermittlung und -gestaltung. Er plädiert für
eine intensivere Auseinandersetzung mit solchen Veränderungsprozessen.
Dabei verhandelt er zwei miteinander verwobene Phänomene. Das erste
betrifft die digitale Verwandlung von Wörtern bzw. Wortketten in eine
Warenform, wie sie den Kern von Googles Geschäftsmodell in einer
\enquote{global linguistic economy} darstellt. Eine klassische
Bezeichnung dafür ist: Kommodifizierung. Der Preis für die neben (oder
in) Suchergebnissen platzierten Anzeigen wird dynamisch berechnet und
versteigert. Bestimmte entweder sehr zweckklare oder eben sehr populäre
Zeichenketten (\enquote{Blumen}, \enquote{Automobile kaufen},
\enquote{Miley Cyrus}) sind in diesem Zusammenhang erwartungsgemäß
teurer und somit möglicherweise für den so genannten
\enquote{linguistischen Kapitalismus} (linguistic capitalism)
wertvoller. Anders als bei klassischer Werbung steht bei diesem weniger
die Aufmerksamkeitsökonomie im Zentrum, sondern etwas, das man als
Ausdrucks- oder Äußerungsökonomie bezeichnen kann (economy of
expression).

Generell gilt: \enquote{Anything that can be named can be associated
with a bid.} (Kaplan, S. 59) Umso entscheidender ist es für die Anbieter
solcher Auktionen, die naheliegend das Ziel einer umfassenden
Kommodifizierung von Sprache verfolgen, ein hohes Verständnis von
Sprache und Sprachgebrauch zu entwickeln und am besten bei Bedarf auch
steuernd eingreifen zu können. Das Verständnis entsteht bei Google
traditionell hauptsächlich aus der statistischen Erfassung und Analyse
von textuellem Handeln bzw. Verhalten und zwar in möglichst vielen
Zusammenhängen (=Kon-Texten) und von möglichst vielen Akteuren. Auf
dieser Basis werden die linguistischen Beziehungen ermittelt, die nach
ihrer Häufigkeit unter anderem bei der Autovervollständigung im
Eingabefenster bei Google oder dem Angebot von alternativen
Zeichenketten bei der Ergebnisdarstellung als zusätzlicher Dienst
erscheinen. An dieser Stelle ist dann auch prinzipiell ein gezielt
lenkender Eingriff in das Sprachverhalten denkbar. Kaplan schreibt:

\begin{quote}
It transforms linguistic material without value (not much bidding on
misspelled words) into a potentially profitable economic resource. When
Google automatically extends a sentence you have started to type, it
does more than save your time, it transforms your expression into one
that is statistically more regular based on the linguistic data it daily
gathers. Even if Google's autocompletion may not be explicitly biased
toward more economically valuable expressions, it nevertheless tends to
transform natural language into more regular, economically exploitable
linguistic subsets. (S. 59f.)
\end{quote}

Anhand des durch die Indexierung von Webseiten, Büchern, Metadaten und
konkreten textuellen Mensch-Maschine-Interaktionen entstandenen Korpus
verfügt Google über ein enormes statistisches Wissen zu Relationen
zwischen Zeichenketten. Außerdem -- was Kaplan nicht erwähnt -- bei
entsprechend nachvollziehbaren Profilen auf der Basis von Gmail oder
Google+ auch über Relationen zwischen dem Zeichen- und Sprachgebrauch
bestimmter und bestimmbarer Akteure, die wiederum als soziales Netzwerk
erfasst und über Metadaten kategorisiert werden können. Die
Grunderkenntnis bestätigt sich auch hier: Der digitale Kapitalismus ist
elementar semiotisch.

In der Wirkung reicht er jedoch weit über den Bereich des Sprachlichen
hinaus. Wenn Kaplan die Frage stellt, wie gerade in der Wechselwirkung
von originär menschlichem Input (primary resources) und algorithmisch
erzeugten bzw. modifizierten Texten (secondary resources) bestimmte
Sprachverzerrungen und -veränderungen entstehen, betrachtet er freilich
nur einen Ausschnitt aus der Bandbreite möglicher Konsequenzen.
Möglicherweise sind nämlich die simplen adaptiven Verfahren der
Autovervollständigung, die eventuell unseren Sprachgebrauch beeinflussen
genauso wenig wie eine damit einhergehende denkbare
\enquote{Kreolisierung} der Sprache die am dringlichsten zu
analysierenden Auswirkungen des derzeitigen Quasimonopols, das Google
auf dem digitalen \enquote{multilingual lingustic market} innehat.

So interessant sich das Konzept des \enquote{linguistic capitalism}
präsentiert, so sehr vernachlässigt es in der Darstellung bei Kaplan die
Spannweite der Effekte der digitalen Kodifizierung der menschlichen
Lebenswelt.

Eine aktuelle \enquote{Declaration on Digital Capitalism} der
\enquote{International Necronautical Society} (einem Projekt des
Schriftstellers Tom McCarthy und des Philosophen Simon Critchley)
rotiert das Phänomen in einer leider etwas aufgesetzten Art in einem
weiter fassenden Radius, wenn es in ihr, in Rückgriff auf Michel De
Certeaus Vor-Internet-Buch \enquote{The Practice of Everyday Life}
(1980/1984) heißt:

\begin{quote}
We inhabit a world of endless and inescapable codings and notations, a
world whose central currency is legibility. The exercise of power is a
\enquote{scriptural enterprise.} Citizen agents, whom he {[}de
Certeau{]} dubs consumers rather than individuals, orchestrate small
parole acts through the language of capitalist culture, whose matrix of
legibility conspires to capture and decode even the most idiosyncratic.
What might escape this matrix? Nothing, not even bodies, since all
bodies are already seized hold of and written, transformed into code.
(McCarthy, Critchley, 2014, S. 259)
\end{quote}

Die Leitidee des digitalen (Sprach-)Kapitalismus ist demzufolge
naheliegend die Auflösung von allen dafür tauglichen Phänomenen in
maschinell verarbeitbare Zeichen und deren Prozessierung durch bestimmte
Algorithmen. Das Ziel dabei ist, wie bei jedem Marktgeschehen, die
möglichst weitreichende Kontrolle dessen, was möglich wird und damit
verbunden die Reduzierung von Unsicherheit. Was in den 1980ern und
vielleicht schon eher Idee und Ansatz war (man denke an die Konzepte des
so genannten Wissensmanagements), ist heute sehr weitreichend technisch
umsetzbar. Die digitale Kodifizierung ermöglicht so die Kommodifizierung
von Sinneinheiten bzw. Informationsobjekten, die sich der Verwandlung in
marktförmige Produkte zuvor entzogen. Grundlage dieser Ökonomie ist
immer die Kombination aus eindeutiger Benennbar- und Lesbarkeit
(legibility) des Bezugscodes und die statistische Erfassung von
Relationen zwischen digitalen Objekten, zum Beispiel kategorisiert und
vernetzt in der Dreiheit Akteur, Handlung, Ereignis.

Google ist in dieser Totalisierung der digitalen Kontrolle und
Governance bei weitem nicht der einzige Akteur. Aber dadurch, dass es
Google sehr früh gelang diese im Nachgang fast bestürzend einfachen
Grundideen der Verknüpfung von Objekt- und Zeichenrelationen in eine
massentaugliche und damit auch vermarktbare Dienstleistung zu
verwandeln, ist es bis heute der zentrale Türhüter und damit
wahrscheinlich das zentrale Machtzentrum der an dieser Stelle
erstaunlich wettbewerbsarmen digitalen Ökonomie.

Frederic Kaplan (2014) Linguistic Capitalism and Algorithmic Mediation.
In: Representations, Vol. 127, No. 1 (Summer 2014), S. 57-63. URL:
\url{http://www.jstor.org/stable/10.1525/rep.2014.127.1.57}

Thomas McCarthy, Simon Critchley (2014) Declaration on Digital
Capitalism. In: Artforum International. October 2014, S. 254-259

\emph{Überarbeitete Fassung eines Beitrags im LIBREAS-Tumblr:}

\url{http://libreas.tumblr.com/post/99407407976/linguistic-capitalism}

%autor

\end{document}