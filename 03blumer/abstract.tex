Gentrification -- a process of replacement of a poorer population in an
urban neighborhood with a richer one and the change of the looks of this
respective neighborhood -- has become a widespread topic of societal
debates in recent years. This process is linked to public art and
cultural activities, and sometimes triggered with projects of urban
revitalization by the respective cities. Public libraries are part of
this process and they are used in the concepts of urban revitalization
as well as institutions for public culture. This puts them in an uneasy
position: thus, libraries also become part in processes of the repulsion
of socially vulnerable groups. The text will discuss the current
position of public libraries in respect to gentrification, using some
incidents in the city of Lausanne, Switzerland, as a case study.
