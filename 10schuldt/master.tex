\documentclass[a4paper,
fontsize=11pt,
%headings=small,
oneside,
numbers=noperiodatend,
parskip=half-,
bibliography=totoc,
final
]{scrartcl}

\usepackage{synttree}
\usepackage{graphicx}
\setkeys{Gin}{width=.4\textwidth} %default pics size

\graphicspath{{./plots/}}
\usepackage[ngerman]{babel}
\usepackage[T1]{fontenc}
%\usepackage{amsmath}
\usepackage[utf8x]{inputenc}
\usepackage [hyphens]{url}
\usepackage{booktabs} 
\usepackage[left=2.4cm,right=2.4cm,top=2.3cm,bottom=2cm,includeheadfoot]{geometry}
\usepackage{eurosym}
\usepackage{multirow}
\usepackage[ngerman]{varioref}
\setcapindent{1em}
\renewcommand{\labelitemi}{--}
\usepackage{paralist}
\usepackage{pdfpages}
\usepackage{lscape}
\usepackage{float}
\usepackage{acronym}
\usepackage{eurosym}
\usepackage[babel]{csquotes}
\usepackage{longtable,lscape}
\usepackage{mathpazo}
\usepackage[flushmargin,ragged]{footmisc} % left align footnote

\usepackage{listings}

\urlstyle{same}  % don't use monospace font for urls

\usepackage[fleqn]{amsmath}

%adjust fontsize for part

\usepackage{sectsty}
\partfont{\large}

%Das BibTeX-Zeichen mit \BibTeX setzen:
\def\symbol#1{\char #1\relax}
\def\bsl{{\tt\symbol{'134}}}
\def\BibTeX{{\rm B\kern-.05em{\sc i\kern-.025em b}\kern-.08em
    T\kern-.1667em\lower.7ex\hbox{E}\kern-.125emX}}

\usepackage{fancyhdr}
\fancyhf{}
\pagestyle{fancyplain}
\fancyhead[R]{\thepage}

%meta
%meta

\fancyhead[L]{K. Schuldt \\ %author
LIBREAS. Library Ideas, 26 (2014). % journal, issue, volume.
\href{http://nbn-resolving.de/urn:nbn:de:kobv:11-100222695
}{urn:nbn:de:kobv:11-100222695}} % urn
\fancyhead[R]{\thepage} %page number
\fancyfoot[L] {\textit{Creative Commons BY 3.0}} %licence
\fancyfoot[R] {\textit{ISSN: 1860-7950}}

\title{\LARGE{Queer as Archive. Rezension zu: Alana Kumbier (2014): Ephemeral Material - Queering the Archive. (Gender and Sexuality in Information Studies ; 5). Sacramento, CA : Litwin Books}} %title %title
\author{Karsten Schuldt} %author

\setcounter{page}{77}

\usepackage[colorlinks, linkcolor=black,citecolor=black, urlcolor=blue,
breaklinks= true]{hyperref}

\date{}
\begin{document}

\maketitle
\thispagestyle{fancyplain} 

%abstracts

%body
\emph{Ephemeral Material} ist eine Reflexion über die Möglichkeiten und
vor allem die Bedeutungen von archivalischen und bibliothekarischen
Tätigkeiten für und über Subkulturen an Schnittstellen zwischen
Selbstorganisation, Kunstprojekten und Institutionen.

Die im Mittelpunkt stehende Subkultur ist die der Drag Kings und Queens
im US-a\-me\-ri\-ka\-nischen Kontext, insbesondere in und um New Orleans; der
Anspruch des Buches geht über diese Subkultur hinaus. An allen im Buch
beschriebenen Unternehmungen ist oder war die Autorin aktiv beteiligt.
Sie greift also auf Insiderwissen zurück, welches so selbstverständlich
durch diese Perspektive geprägt ist. Dabei nimmt sie aber mehrere Rollen
ein. Sie war Teil der Drag\-Sub\-kultur in New Orleans, ist akademisch
ausgebildete Bibliothekarin, die jetzt auch in einer Bibliothek
arbeitet. Sie ist mit Personen mit ähnlicher Biographie vernetzt und
nimmt weiterhin teil an Projekten der Subkultur. Es ist dabei
anzumerken, dass in dieser Subkultur eine aktive Teilnahme -- also nicht
nur als Zuschauende, sondern auch als Person, die in Bands auftritt, in
Projekten mitarbeitet, in kleinen Publikation schreibt und so weiter --
die Normalität darstellt.

In der Drag-Subkultur geht es bekanntlich immer um das Performen, Leben
und gleichzeitige Hinterfragen von Geschlecht und Geschlechterrollen. Es
ist eine Subkultur, die per se eine politische Position einnimmt, auch
wenn die Beteiligten das nicht unbedingt immer wollen. Alana Kumbier
nimmt dieses kritische Potential im Titel und Vorwort des Buches als
Anspruch auf. Sie will die dominierenden Diskurse im Archiv- und
Bibliotheksbereich durchschreiten, das heisst nicht unbedingt direkt
angreifen und widerlegen, sondern aus unerwarteten Positionen heraus
hinterfragen und verändern. Für solche Vorhaben hat sich der Begriff des
durchqueerens -- inklusive des Adjektivs queer -- etabliert. Dieses
Vorhaben ist übereinstimmend mit dem Anspruch der Subkultur, aber auch
der akademischen Debatte um Queerness. Grundsätzlich aber sind die
Einsichten des Buches für andere Subkulturen sowie archivalische und
bibliothekarische Projekte für sie oder in ihnen ebenso relevant. Dies
gilt nicht nur für Subkulturen, deren Ästhetik und Anspruch der
Drag-Subkultur ähnlich ist, wie Riot Grrrl, anderen feministischen oder
schwul-lesbischen Subkulturen oder unterschiedlichen Punk-Subkulturen.
Auch Subkulturen, die sich um andere Ausdrucksformen -- zum Beispiel
Graffiti --, andere Themen -- zum Beispiel Trekkies oder Live Action
Role Players -- oder politische Inhalte -- zum Beispiel Antifaschismus
oder Antirassismus -- gruppieren, werden ähnlich funktionieren. Hierin
liegt aber auch der Vorteil des Buches für die gesamte Debatte im
Bibliotheks- und Archivbereich. Eine Subkultur, die zudem sehr sichtbar
ist, wenn Sie sich bemerkbar macht, wird als Beispiel für Fragen
genutzt, die sich auch für andere Subkulturen stellen. Ob der Anspruch
der Queerness eingehalten wird, ist in dieser Leseweise eine Frage, die
berechtigt ist, aber sich für Bibliotheken und Archive weniger stellt.

\subsubsection{Die Einrichtung ist für die Community
da}\label{die-einrichtung-ist-fuxfcr-die-community-da}

Hauptargument des Buches ist, dass Community-Archives -- die sehr
unterschiedliche Formen annehmen und nicht unbedingt immer nach
Bibliothek oder Archiv zu unterteilen sind -- für die jeweilige
Community Funktionen übernehmen, insbesondere die Identität der
Community und die eigene Geschichtsschreibung unterstützen und
gleichzeitig als soziale Zentren selbiger funktionieren können -- aber
gleichzeitig nicht unbedingt wie Bibliotheken oder Archive funktionieren
müssen, oft sogar übliche bibliothekarische und archivalische
Tätigkeiten, Ziele und Standards negieren. Zudem wird den
Community-Archives von der Autorin eine pädagogische Funktion
zugeschrieben, die in erster Linie in die Community hinein wirkt und zum
Beispiel das Verständnis der Geschichte der Subkultur mitsamt ihren
Debatten und sich ändernden Strukturen vermittelt, und erst in zweiter
Linie aus der Subkultur heraus. Dieses Argument wird, wie schon bemerkt,
vor allem aus der persönlichen Erfahrung der Autorin entfaltet, aber mit
Rückgriff auf breitere Literatur über ähnliche Projekte.

Dabei wird den Community-Archives eine direkte Verbindung zu den
politischen Zielen der jeweiligen Community zugeschrieben, was sie aber
gleichzeitig davon abhängig macht:

\begin{quote}
Without community support and involvement, the archives wouldn't grow,
necessary work wouldn't be accomplished, and the archives wouldn't
reflect the constituencies and experiences they seek to document. In
this way, these archival projects align with other queer projects, to
transform dominant, oppressive social and political orders. (Kumbier
2014, S. 8)
\end{quote}

\begin{quote}
Though grassroots archives may lack financial and labor resources, they
posses an important resource that conventional archives often don't have
-- that of community knowledge. (Kumbier 2014, S. 27)
\end{quote}

Das Buch besteht aus zwei Teilen. Im ersten bespricht die Autorin anhand
von zwei Filmen das Bild des Archivs aus jeweils sehr spezifischen
Blickwinkeln, um die Grenzen von Archiven auszuloten. Im zweiten Teil
widmet sie sich anhand von Projekten, an denen sie selber beteiligt war,
der Beziehung von Community und Community-Archives.

Der erste Film stellt, anhand einer Suche nach einer fiktiven
afroamerikanischen, lesbischen Schauspielerin der 1920er bis 1940er
Jahre einerseits die Konstruktion von Geschichte als counter-narrative
dar -- in dem Sinne, dass es die gesuchte Schauspielerin nie gab, dass
aber die Möglichkeit einer Konstruktion derer Geschichte schon einen
Diskursraum eröffnet. Andererseits gibt sie einen satirischen Einblick
in feministische und andere Community-Archives. Auch der zweite Film
verhandelt die Konstruktion von unterschiedlichen Geschichten. Hier die
einer kleinwüchsigen Jüdin, die das KZ überlebte und heute ein anderes
Bild von Josef Mengele vertritt, als ihre Bekannte, die als Historikerin
auf die Suche nach einem Film geht, welchen Mengele von der Familie der
Jüdin angefertigt hatte. Dieser Film soll auf Wunsch der KZ-Überlebenden
vernichtet werden, da er die gesamte Familie als Untersuchungsobjekte
präsentiert. Im dazugehörigen Dokumentarfilm wird die Recherche nach
diesem Film inklusive der Kommunikation zwischen Forschender und
Überlebender gezeigt, die sich ebenfalls beständig um Fragen der
Konstruktion von Geschichte und Identität dreht. Gleichzeitig bleiben,
auch da der Film nicht gefunden wird, Hauptfragen offen. Darf solch ein
Dokument zerstört werden? Wer bestimmt dann über wessen Geschichte, wenn
über diese Frage entschieden wird? Dass die Forschende selber anders
befähigt ist und die Archive nur mithilfe Ihres Assistenzteams nutzen
kann, eröffnet eine weitere Ebene, da der Dokumentarfilm zeigt, wie sehr
die Archive auf \enquote{normale} Forschende hin eingerichtet sind und
in gewisser Weise ausschliessend wirken.

Beide Filme und die Diskussionen derselben durch Kumbier bewegen sich
erkennbar im Rahmen postmoderner Theorien. Geschichte wird zum Beispiel
als verhandelbar begriffen, als Ergebnis von Interpretationen, Kämpfen,
Deutungen und Erzählungen. Mit der Akzeptanz dieser Theorien ändern sich
auch die Bedingungen und Aufgaben der genutzten Archive und
Bibliotheken. Sie werden Teil dieser Verhandlungen von Geschichte und
Erzählungen. Insoweit ist es konsequent, wenn die Autorin eine Bedeutung
der Archive und Bibliotheken bei der Konstitution von Subkulturen
ableitet.

Dabei ist diese Bedeutung gegenseitig. Funktionierende
Community-Archives existieren für die Community und tragen zu ihr bei,
funktionieren aber auch nur durch die Community, welche sie
grösstenteils am Leben erhalten. Solche Bibliotheken und Archive sind,
so Kumbier, immer ein \enquote{work of love}, also ein Ort freiwilligen
Engagements, das von Spenden im Sinne von Arbeitszeit, Geld und
Objektiven lebt. Sie haben einen immateriellen Wert für die Communities,
sammeln dafür aber auch vieles, was in traditionellen Bibliotheken und
Archiven nicht gesammelt wird, sondern nur durch die Bedeutung für die
jeweilige Community selber -- im Buch am Beispiel von Kostümen von
Drag-Artists besprochen -- relevant werden. Oft ist es zudem schwierig,
die jeweiligen Einrichtungen als Bibliothek oder Archiv zu beschreiben.
Zumeist weisen sie Eigenheiten beider Formen von
Informationseinrichtungen auf und gehen über sie hinaus.

\subsubsection{Archiving Drag}\label{archiving-drag}

Im zweiten Teil des Buches diskutiert die Autorin eingehend
unterschiedliche Projekte, die sich alle um die Dokumentation der
Drag-Subkultur im Süden der USA, insbesondere seit Mitte der 1990er
Jahre, bemühten. Wie schon erwähnt ist diese Setzung durch die
Biographie der Autorin bestimmt, die in dieser Zeit Teil jener Subkultur
wurde, während sie in New Orleans studierte.

Das erste Projekt scheiterte. Es wurde von einer interessierten
Bibliothekarin an einer der lokalen Universitäten -- die ebenfalls in
der Subkultur engagiert ist -- und der Autorin gestartet und sollte die
Drag-Subkultur mithilfe der Subkultur selber sammeln. Angesiedelt war
das Projekt an der Universitätsbibliothek, die einen Auftrag zur
Sammlung von Materialien von Kulturen und Subkulturen ihrer Umgebung
hat. Insoweit war eine institutionelle Anbindung gegeben. Gleichzeitig
handelte es sich bei den Initiatorinnen um Aktivistinnen, die für und
mit der Subkultur zusammen eine Sammlung gestalten wollten. Sie griffen
zum Beispiel darauf zurück, szenetypische Flyer für das Projekt an
szenetypischen Orten zu verteilen. Ihr Versuch lief aber ins Leere. Nur
wenige Objekte wurden gesammelt, nur wenige Interviews -- die Teil der
Sammlung werden sollten -- konnten geführt werden. In der Reflexion
dieses Scheiterns stellt die Autorin die These auf, dass zwar ihre
Mitaktivistin und sie das Gefühl hatten, es wäre notwendig, die
Geschichte der Drag-Subkultur der Umgebung zu erhalten, aber dieses
Gefühl nicht von dieser Community geteilt wurde. Gleichzeitig bemerkt
sie, dass in der Szene selber die Idee hinter dem Projekt nie richtig
klar wurde, egal, wie viele Flyer verteilt wurden.

Ein weiteres gescheitertes Projekt stellte die Dokumentation des
International Drag King Community Extravaganza dar. Dieses jährliche, je
ein Wochenende dauernde Treffen der Drag-Subkultur mit politischen und
kulturellen Veranstaltungen startete 1999 in New Orleans und wanderte
anschliessend über den nordamerikanischen Kontinent. Kumbier war seit
dem zweiten Extravaganza aktiv beteiligt und versuchte mehrfach
anzuregen, dass diese Veranstaltungen dokumentiert werden. Es gab von
ihr und anderen Versuche, dies über Aufrufe für Sammlungen, über ein
Blog und ein Buchprojekt zu realisieren, die allesamt wenig Rücklauf
erhielten. Dies führte die Autorin auch auf die Struktur der
Veranstaltung zurück: Eine lokale Vorbereitungsgruppe organisiert ohne
grosse Anleitung oder Unterstützung, zumal ohne wirkliche materiellen
Mittel, ein internationales Treffen für mehrere hundert Personen und ist
offenbar anschliessend nicht mehr gewillt oder in der Lage, für eine
Dokumentation zu sorgen. Gleichzeitig aber betont die Autorin, dass
wieder die Szene selber nicht unbedingt die Notwendigkeit verspürte, die
Geschichte dieser Veranstaltung zu überliefern.

Beim achten Extravaganza führte die Autorin einen Workshop im Rahmen der
Veranstaltung durch, bei dem es sich darum drehte, überhaupt die Idee
einer solchen Dokumentation anzuregen. Dieser Workshop -- dem weitere
folgten -- war erfolgreich. Ein wechselndes Team erstellte in den darauf
folgenden Extravaganzas Sammlungen, Ausstellungen und Fanzines. Diese
Sammlung wird privat gelagert -- etwas, was für institutionalisierte
Archive und Bibliotheken undenkbar wäre. Aber sie wird von der Community
gepflegt und rezipiert.

Basierend auf dieser Erfahrung stellt die Autorin vier mehr oder minder
genaue Grundregeln für Community-Archives auf:

\begin{enumerate}
\def\labelenumi{\arabic{enumi}.}
\item
  advocate for archives, das heisst, dass den jeweiligen Communities
  vermittelt wird, welchen Sinn eine Sammlung haben kann,
\item
  ask the community instead of assuming, sowohl dazu, ob sie eine
  Sammlung sinnvoll finden würde, als auch welche Form von Sammlung mit
  welchen Inhalten und Aufgaben,
\item
  lasst die Aktiven selber entscheiden, was gesammelt und dokumentiert
  werden soll, dies motiviert die Selbstdokumentation und
\item
  ermöglicht die direkte Mitarbeit.
\end{enumerate}

Auch bei dieser Liste wird ersichtlich, dass die Community-Archives
anders funktionieren, als herkömmliche.

Zwei weitere Projekte, über welche die Autorin berichtet, waren direkt
im ersten Anlauf erfolgreich. Das erste war eine Performance, in der
eine gut vernetzte Aktivistin der Drag-Szene zwei Tage lang alle
Menschen, die vorbei kamen -- zumeist solche, die mit ihr irgendwie
bekannt waren -- an den Wänden einer Galerie die Photos ihres Lebens
ordnen liess, sowohl chronologisch als auch thematisch. Diese gemeinsame
Arbeit erstellte eine temporäre Sammlung, die selbstverständlich nur
durch die Kontextualisierung durch die Anwesenden Sinn erhielt. Für
Kumbier ist dies ein Teil der Arbeit von Community-Archives. Obwohl
tendenziell schneller vergänglich als herkömmliche Einrichtungen,
stellen sie doch einen zentralen Ort für die Beteiligten dar.

Beim zweiten Projekt handelt es sich um das Queer Zine Archive Project,
ein privat getragenes und finanziertes Projekt zur Sammlung,
Katalogisierung und Digitalisierung von Fanzines -- selbstpublizierter
Hefte von Aktivistinnen und Aktivisten in Kleinstauflage und oft mit Do
it yourself-Ästhetik --, die sich in irgendeiner Weise als Queer
verstehen. Fanzines sind qua ihrer prekären Existenz -- immer abhängig
von der Lust und dem Möglichkeiten der Herausgebenden, meist einer
Person -- davon bedroht, zu verschwinden. Durch die Archivierung
derselben wird es möglich, die Geschichte der queeren Subkulturen besser
nachzuvollziehen. Interessanter als dieser Einblick in die Subkultur ist
an dem Projekt allerdings die Struktur im Hintergrund. Das Projekt ist
in einem Raum einer Privatwohnung angesiedelt, wird vom dort lebenden
Paar und einigen weiteren Aktivistinnen und Aktivisten betreut.
Hauptentscheidungsort sind gemeinsame Abendessen. Es lässt
Praktikantinnen und Praktikanten bei sich arbeiten. Gleichzeitig lässt
es, zumindest vom Anspruch her, die Publizierenden der Fanzines selber
entscheiden, wie die einzelnen Zines erinnert werden sollen. Ein Teil
der Metadaten zu den Digitalisaten werden von den Publizierenden
erstellt, ebenso entscheiden diese, ob ihr Zine als queeres Zine gilt
oder nicht. Insoweit reflektiert das Projekt auch die Szene selber:
Prekär, abhängig von \enquote{work of love}, gleichzeitig aber auch
immer wieder integrierend.

\subsubsection{Queering the archive?}\label{queering-the-archive}

Das Buch ist in seiner biographischen Offenheit interessant. Wer jemals
länger Teil einer Subkultur gewesen ist, wird vieles wiedererkennen.
Allerdings ist die Autorin mit dem Anspruch angetreten, die Grenzen und
Regeln der herkömmlichen Archive und Bibliotheken zu \enquote{queeren}.
Es stellt sich die Frage, ob ihr das gelungen ist. In gewisser Weise
liest sich das Buch wie ein Hinweis darauf, was herkömmliche
Einrichtungen anders machen können, wenn sie sich auf Subkulturen
einlassen wollen. Es zeigt auch, warum bestimmte Projekte mit
Subkulturen scheitern. Grundsätzlich scheint es zu einer grösseren
Empathie aufzurufen und argumentiert dafür, die Grundannahmen
postmoderner Theorien als Folie für das Verstehen von Community-Archives
zu nutzen.

Aber ist das queer? Es ist eher postmodern, wobei es richtig ist, dass
postmoderne Theoriebildung zumeist anhand von queere Subkulturen, wie
auch die Drag-Subkultur, betrieben wird, auch weil diese Kulturen an den
Rändern von vorgeblich festen Kategorien wie dem Geschlecht angesiedelt
sind.

Bedeutsamer ist vielleicht, dass die Autorin für einen anderen Umgang
mit Subkulturen und ihren Sammlungen, Archiven, Bibliotheken
argumentiert. Solange eine Subkultur eine Einrichtung dieser Art
benötigt, wird sie sie offenbar selber schaffen und erhalten. Dann
erfüllt sie für die Subkultur wichtige Funktionen. Manchmal hört dieser
Zustand auf, weil Subkulturen verschwinden, sich entwickeln, andere
Interessen ausprägen. Dann können (herkömmliche) Archive und
Bibliotheken gefragt sein, solche Sammlungen zu übernehmen. Dies
widerspricht in gewisser Weise dem gängigen Selbstbild von Archiven und
Bibliotheken. Ob es dafür der Drag-Subkultur als Beispiel benötigt
hätte, wird nicht ersichtlich. Gewiss hätte eine solche Argumentation
auch anhand anderer Subkulturen vorgebracht werden können, wobei
selbstverständlich die angebrachten Beispiele im zweiten Teil eine gute
Argumentationsbasis bieten, insbesondere weil sich die Autorin nicht
scheut, das Scheitern von Projekten einzugestehen und zu reflektieren.

Am Ende des Buches steht das Gefühl, dass es inhaltlich vor allem um das
Verstören des herkömmlichen Selbstverständnisses von Archiven und
Bibliotheken geht -- etwas, dass die Entwicklung der Archiv- und
Bibliothekswesen vorantreiben kann.

%autor
\begin{center}\rule{0.5\linewidth}{\linethickness}\end{center}

\textbf{Dr.~Karsten Schuldt} is a research fellow at the Swiss Institute
of Information Science at the University of Applied Sciences in Chur and
editor of LIBREAS. Library Ideas. Both live and work in Berlin, Chur,
Geneva and Lausanne.

\end{document}