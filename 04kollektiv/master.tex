\documentclass[a4paper,
fontsize=11pt,
%headings=small,
oneside,
numbers=noperiodatend,
parskip=half-,
bibliography=totoc,
final
]{scrartcl}

\usepackage{synttree}
\usepackage{graphicx}
\setkeys{Gin}{width=.6\textwidth} %default pics size

\graphicspath{{./plots/}}
\usepackage[ngerman]{babel}
\usepackage[T1]{fontenc}
%\usepackage{amsmath}
\usepackage[utf8x]{inputenc}
\usepackage [hyphens]{url}
\usepackage{booktabs} 
\usepackage[left=2.4cm,right=2.4cm,top=2.3cm,bottom=2cm,includeheadfoot]{geometry}
\usepackage{eurosym}
\usepackage{multirow}
\usepackage[ngerman]{varioref}
\setcapindent{1em}
\renewcommand{\labelitemi}{--}
\usepackage{paralist}
\usepackage{pdfpages}
\usepackage{lscape}
\usepackage{float}
\usepackage{acronym}
\usepackage{eurosym}
\usepackage[babel]{csquotes}
\usepackage{longtable,lscape}
\usepackage{mathpazo}
\usepackage[flushmargin,ragged]{footmisc} % left align footnote

\usepackage{listings}

\urlstyle{same}  % don't use monospace font for urls

\usepackage[fleqn]{amsmath}

%adjust fontsize for part

\usepackage{sectsty}
\partfont{\large}

%Das BibTeX-Zeichen mit \BibTeX setzen:
\def\symbol#1{\char #1\relax}
\def\bsl{{\tt\symbol{'134}}}
\def\BibTeX{{\rm B\kern-.05em{\sc i\kern-.025em b}\kern-.08em
    T\kern-.1667em\lower.7ex\hbox{E}\kern-.125emX}}

\usepackage{fancyhdr}
\fancyhf{}
\pagestyle{fancyplain}
\fancyhead[R]{\thepage}

%meta
%meta

\fancyhead[L]{Bibliothekskollektiv \\ %author
LIBREAS. Library Ideas, 26 (2014). % journal, issue, volume.
\href{http://nbn-resolving.de/urn:nbn:de:kobv:11-100219254
}{urn:nbn:de:kobv:11-100219254}} % urn
\fancyhead[R]{\thepage} %page number
\fancyfoot[L] {\textit{Creative Commons BY 3.0}} %licence
\fancyfoot[R] {\textit{ISSN: 1860-7950}}

\title{\LARGE{Anarchistische Bibliothek \& Archiv Wien}} %title %title
\author{Bibliothekskollektiv Anarchistische Bibliothek \& Archiv Wien} %author

\setcounter{page}{}

\usepackage[colorlinks, linkcolor=black,citecolor=black, urlcolor=blue,
breaklinks= true]{hyperref}

\date{}
\begin{document}

\maketitle
\thispagestyle{fancyplain} 

%abstracts

%body
Die Anarchistische Bibliothek in Wien gibt es seit 2010 und wird von
einer Handvoll Leuten unentgeltlich betrieben. Sie existiert sowohl real
in den angemieteten Räumlichkeiten im Erdgeschoß im Hinterhof eines
Gründerzeithauses im 8. Wiener Gemeindebezirk als auch im Netz unter
\url{http://a-bibliothek.org}.

Beide Orte sind uns wichtig, denn wir sehen unsere Bibliothek als
sozialen und kommunikativen Treffpunkt für Anarchist\_innen, die in Wien
oder in anderen Orten auf der Welt leben und hier mit uns kommunizieren.
Dabei dient unser Auftritt im Netz nicht nur der Selbstdarstellung und
Kommunikation nach außen, sondern auch als Service für Interessierte.
Unser Buchbestand von circa 2.500 Büchern ist bis jetzt zur Hälfte
katalogisiert und soll in Zukunft vollständig systematisiert sein.

Im Digitalisierungsprojekt, dessen Fortführung allerdings von
vorhandenem Geld und Zeit abhängig ist, werden Zeitschriften, Broschüren
und Bücher aus der Geschichte des Anarchismus elektronisch aufbereitet
und können im Internet kostenfrei abgerufen werden. Die Digitalisate
werden angefertigt, um einerseits einem Text eine längere Lebensdauer
als dem Papier, auf dem er gedruckt ist (was sich aber erst
herausstellen wird) zu geben und andererseits, um die Zugänglichkeit zu
verbessern. Elektronisch aufbereitet heißt: Die jeweiligen Druckwerke
werden digitalisiert und dann mit OCR bearbeitet. Folglich werden die
eingescannten Seiten nicht als Bild auf die Webseite gestellt, sondern
als durchsuchbares Textdokument. Weil wir nicht nur den Inhalt
zugänglich machen, sondern auch die Originalansicht erhalten wollen,
geschieht das in einem Zweischichtverfahren: Zu sehen ist die Seite im
Originallayout und unsichtbar, quasi dahinter, ist das Textdokument, das
gänzlich durchsuchbar ist. Die OCR-Bearbeitung ist recht zeitaufwendig,
da sich bei der Umwandlung in ein Textdokument -- je nach Qualität der
Vorlage -- etliche bis sehr viele Fehler einschleichen, die danach
händisch wieder ausgebessert werden müssen. Aber wir finden, wenn einem
Digitalisat schon all die sinnlichen Eindrücke, die mensch sonst beim
Lesen eines alten Schriftstückes hat (wie greift sich das Papier an, wie
riecht es, das Rascheln beim Umblättern\ldots{}) fehlen, dann sollten
doch auf der anderen Seite die Möglichkeiten, die durch die
Digitalisierung hinzu gekommen sind, auch genutzt werden. Unter
\enquote{Digitalisierte Zeitschriften und Bücher} finden sich zum
Beispiel die vollständige Ausgabe der 1907--1914 von Pierre Ramus in
Österreich herausgegebenen Zeitschrift \enquote{Wohlstand für Alle}, die
Broschüre von Madeleine Vernets \enquote{Die freie Liebe} (1920) oder
das Buch von Joseph Peukert \enquote{Gerechtigkeit in der Anarchie} --
zur Zeit allerdings nicht einsehbar, da uns der Server-Platz fehlt.

Aufgrund unserer politischen Überzeugung und in diesem Sinne aus einer
anarchistischen Perspektive betrachten wir im Folgenden die für uns
wichtigen Aspekte bezüglich unserer Bibliothek und der Vermittlung an
interessierte Leser\_innen. Weiter handelt es sich um Punkte, die
Bibliotheken einer politischen beziehungsweise sozialen Bewegung eigen
sind und sich dadurch von herkömmlichen unterscheiden.

Anarchismus bedeutet ganz allgemein formuliert eine Gegner\_innenschaft
gegenüber Herrschaft. Dies betrifft sowohl den Staat als auch andere
Formen der Unterdrückung des Menschen. Als politische (Arbeiter\_innen)
Bewegung ist er in der 2. Hälfte des 19. Jahrhunderts entstanden. Es gab
immer wieder Repressionsphasen gegen Anarchist\_innen, unzählige
Zeitschriften mit zensierten Stellen und Lücken; wir wissen, wie
aufwendig und schwierig die Agitation und das Verbreiten anarchistischer
Schriften immer wieder war, dass Bücher verboten wurden und werden und
dass die Gegenwart zeigt, wie schnell politische oder soziale Bewegungen
in den Fokus staatlicher Überwachung und Repression gelangen können.
Weniger dramatisch und einfach formuliert, wollen wir selbst bestimmen,
welche Daten und Informationen in der Bibliothek bleiben und welche nach
außen gehen.

\section*{Sicherheit}\label{sicherheit}

Wir sehen es deshalb als unsere Verantwortung, mit den Daten der
Nutzer\_innen sorgsam umzugehen. Dies bedeutet, wir schützen sie und
benutzen Verschlüsselungsprogramme für die Daten auf unseren Computern,
USB-Sticks und den internen Netzwerk. Um den Email-Verkehr sicher zu
gestalten, bieten wir auf unserer Homepage die Möglichkeit an, per
PGP-Verschlüsselungsprogramm mit uns zu kommunizieren. Auf sonstige
Firewalls oder Virenschutz verzichten wir, denn Open Source heißt auch
weniger Viren.

Zudem gibt es eine umgekehrte Form der Sicherheit: Jene, die auf
Vertrauen beruht. Bibliotheksnutzer\_innen brauchen sich nicht
ausweisen, um sich Bücher ausleihen zu können und werden auch sonst
nicht kontrolliert. Wer Bücher ausleihen möchte, sagt uns (s)einen Namen
und gibt eine Kontaktmöglichkeit bekannt. Die Person vereinbart, das
Buch innerhalb der Verleihfrist von vier Wochen zurückzubringen oder
verlängert die Ausleihfrist. Ob es Bücher gibt, die seitdem nicht mehr
in die Bibliothek zurückgebracht wurden? Ja, gibt es (und wir vermissen
sie).

Jedoch dieses Vertrauen, das wir den Nutzer\_innen entgegen bringen, ist
wichtig, um überhaupt erst zu einer \enquote{freien Vereinbarung}, einem
ebenfalls wichtigen Element in der anarchistischen Ideenwelt, zu
gelangen. Womit wir gedanklich auch schon in die Nähe des nächsten
Punktes rücken würden.

\section*{Open Source}\label{open-source}

Die Computer in unserer Bibliothek laufen ausschließlich auf Linux und
wir verwenden das Koha Open Source-Bibliothekssystem. Wir sind auch
gerade dabei, unseren Buchbestand nach Kategorien wie AutorIn, Titel,
Schlagwörtern, Coverfoto et cetera zu katalogisieren. So befindet sich
Kropotkins \enquote{Gegenseitige Hilfe in Tier und Pflanzenwelt} sogar
inklusive Inhaltsverzeichnis und Probekapitel bereits in unserem
Online-Katalog. Jedoch birgt diese Software für uns noch einige Tücken.
Deshalb suchen wir Menschen, die uns bei der technischen Umsetzung
unterstützen und beraten können. Denn wir sind weder Computer-,
Programmier- noch sonst welche Expert\_innen, aber wir lernen und
versuchen unser (technisches) Wissen in der Gruppe weiterzugeben. Uns
ist es wichtig, so weit wie möglich diese EDV und Netzinfrastruktur ohne
kommerzielle Anbieter\_innen verwenden und anbieten zu können. Wir sehen
den Open Source-Ansatz in Form einer freien Vereinbarung und Kooperation
als zentrales Element, um Wissen und Technik weiter zu entwickeln, als
eine zutiefst anarchistische Herangehensweise. Gerade das Archiv- und
Bibliothekswesen betreffen die Fragen nach langfristiger Verwendbarkeit
und Zugriffsmöglichkeiten von den verschiedenen schriftlichen Beständen.
Wir denken, dass diese nicht in Händen weniger kommerzieller Anbieter
liegen sollen, weshalb wir uns für das Koha-System entschieden haben und
denken, dass eine offene Weiterentwicklung wichtig wäre.

\section*{Selbstorganisation}\label{selbstorganisation}

Wie eingangs erwähnt, ist die Anarchistische Bibliothek ein Projekt, das
von einigen wenigen Leuten aktiv betrieben wird. Dies betrifft sowohl
den Webauftritt, die Digitalisierung von Zeitschriften und Broschüren,
Organisation von Veranstaltungen, Katalogisierung des

Bücherbestandes als auch den wöchentlichen Bibliotheksdienst. Das Plenum
dient als grundsätzliches Entscheidungsgremium und findet regelmäßig
statt. Dort werden die anstehenden Themen besprochen, die Aufgaben
aufgeteilt und Entscheidungen im Konsens getroffen oder, was auch
vorkommen soll, es gibt eben keine gemeinsamen, sondern nur individuelle
Entscheidungen. Auch damit muss eine Gruppe umgehen, will sie nicht auf
eine autoritäre Figur zurückgreifen. Wir wollen dies nicht. Aufgrund der
unterschiedlichen Lebenssituationen ergibt sich ein unterschiedliches
Engagement in der Bibliothek. Diese Unterschiede können zu verschiedenen
Schwierigkeiten und Hierarchien führen. Unsere Aufgabe als
Selbstorganisierende ist es, damit einen Umgang zu finden (Dinge in
eigenen Treffen an- und besprechen), gute Kommunikationsstrukturen
pflegen (den Kalender verwenden), das Wissen weiter geben (interne
Workshops) und so weiter.

Ein berühmter spanischer Anarchist hat einmal in einem Vortrag in Wien
eine Anekdote über das Entstehen von Hierarchien erzählt:

Sie waren gut 30 Leute, hatten ein Kulturzentrum und sich vorgenommen,
jede Person sperrt an einem Tag auf. Der Schlüssel für das Zentrum ging
herum. Wie es sich so ergibt, konnte einer mal nicht, ein anderer sprang
ein und im Laufe der Zeit gab es nur mehr einen, der jeden Tag dort
hinging und das Zentrum aufsperrte. Für die anderen 29 war das zwar eine
Erleichterung. Jedoch die eine Person sah sich nun als Herr des
Schlüssels und sperrte nur mehr auf, wenn er wollte,

denn er war es schließlich, der jeden Tag aufsperrte. Für Abel Paz, so
hieß der Erzähler dieser Anekdote, entstehen so Spezialisierungen und
diese seien der Anfang der Bürokratie und Hierarchie.

Genau eine solche Entwicklung wollen wir mit unserer Form der
Selbstorganisation vermeiden.

\section*{Bibliothek als sozialer
Raum}\label{bibliothek-als-sozialer-raum}

Das Schlüsselbeispiel kann hier noch weiter geführt werden, denn die
Bibliothek soll natürlich auch offen sein. Von Anarchist\_innen für
Anarchist\_innen und all jene Personen, die sich dafür interessieren.
Sie soll als aktiver Bestandteil dieser politischen und sozialen
Bewegung betrachtet und genutzt werden. Dies bedeutet, dass die Nutzung
dieser Bibliothek zu den Öffnungszeiten nicht den herkömmlichen
Bibliotheken entspricht. Sie ist ein Ort der Kommunikation, wo Kaffee,
Wasser oder Bier getrunken werden, wo Leute rauchen und miteinander
quatschen.

Es kommen auch immer wieder Menschen vorbei, die in Wien zu Besuch sind
und die anarchistische Orte aufsuchen beziehungsweise Genoss\_innen
treffen wollen.

Neben den herkömmlichen Bibliothekszeiten, derzeit montags 18 bis 20 Uhr
und nach Vereinbarung, gibt es noch verschiedene Veranstaltungsformate.
Die offenen Lese- und Diskussionsrunden, Lesungen beziehungsweise
Buchpräsentationen, Vorträge und Ausstellungen oder der Tag der offenen
Tür für Lehrlinge einer Bibliotheks- und Archivberufsschule.

Zusammen mit anderen linken*autonomen*radikalen Archiven und
Bibliotheken aus Wien haben wir 2011 die \enquote{Lange Nacht der
Anarchie} und 2012 unter dem Titel \enquote{Radikal hat Bestand} eine
Werbefahrt zur Präsentation der Bibliothekskultur organisiert und
pflegen darüber hinaus internationale Kontakte zu anderen Archiven, wie
dem CIRA (Centre International de Recherches sur l'Anarchisme) in
Lausanne und CIRA Marseille. Denn auch das anarchistische Bibliotheks-
und Archivnetz hat Knoten und die benötigen wir auch. Wir sehen es als
unsere Aufgabe, die Geschichte(n) unserer Bewegung zu sammeln,
weiterzugeben und weiterzuschreiben.

\textbf{Das Gestern im Gedächtnis behalten, das Morgen in die Hand
nehmen!}

%autor

\end{document}
