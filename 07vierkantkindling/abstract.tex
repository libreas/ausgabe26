\emph{Abstract}

Der Artikel analysiert die Ergebnisse des „2014 Cenus on Open Access
Repositories in Germany, Austria and Switzerland" im Hinblick auf die
Institutionen, die in Deutschland Repositorien betreiben. Der 2014
Census basiert auf einer qualitativen Inhaltsanalyse der Webseiten der
Repositorien, einer automatisierten Validierung der über das OAI-PMH
ausgelieferten Metadaten sowie einer Umfrage an die
Repositorienbetreiber im Erhebungszeitraum 6. Januar bis 13. Februar
2014. Die Forschungsdaten des 2014 Census sind auf zenodo.org verfügbar.

\emph{Forschungsdaten}

Forschungsdaten zum ``2014 Census on Open Access Repositories in
Germany, Austria and Switzerland'' unter:
\href{http://doi.org/10.5281/zenodo.10734}{10.5281/zenodo.10734}

\emph{Danksagung}

Unser herzlicher Dank für die gemeinsame Konzeptionierung und
Durchführung des „2014 Census on Open Access Repositories in Germany,
Austria and Switzerland" gilt Dennis Zielke (Computer- und Medienservice
der HU Berlin), Marleen Burger, Anne Lepke, Thomas Maluck, Jessika
Rücknagel, Stephanie van de Sandt und Lisa Theileis (Studierende am
Institut für Bibliotheks- und Informationswissenschaft der HU Berlin)
sowie Friedrich Summann (Universitätsbibliothek Bielefeld) für die
Bereitstellung der BASE-Daten.
