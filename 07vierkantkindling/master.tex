\documentclass[a4paper,
fontsize=11pt,
%headings=small,
oneside,
numbers=noperiodatend,
parskip=half-,
bibliography=totoc,
final
]{scrartcl}

\usepackage{synttree}
\usepackage{graphicx}
\setkeys{Gin}{width=.6\textwidth} %default pics size

\graphicspath{{./plots/}}
\usepackage[ngerman]{babel}
\usepackage[T1]{fontenc}
%\usepackage{amsmath}
\usepackage[utf8x]{inputenc}
\usepackage [hyphens]{url}
\usepackage{booktabs} 
\usepackage[left=2.4cm,right=2.4cm,top=2.3cm,bottom=2cm,includeheadfoot]{geometry}
\usepackage{eurosym}
\usepackage{multirow}
\usepackage[ngerman]{varioref}
\setcapindent{1em}
\renewcommand{\labelitemi}{--}
\usepackage{paralist}
\usepackage{pdfpages}
\usepackage{lscape}
\usepackage{float}
\usepackage{acronym}
\usepackage{eurosym}
\usepackage[babel]{csquotes}
\usepackage{longtable,lscape}
\usepackage{mathpazo}
\usepackage[flushmargin,ragged]{footmisc} % left align footnote

\usepackage{listings}

\urlstyle{same}  % don't use monospace font for urls

\usepackage[fleqn]{amsmath}

%adjust fontsize for part

\usepackage{sectsty}
\partfont{\large}

%Das BibTeX-Zeichen mit \BibTeX setzen:
\def\symbol#1{\char #1\relax}
\def\bsl{{\tt\symbol{'134}}}
\def\BibTeX{{\rm B\kern-.05em{\sc i\kern-.025em b}\kern-.08em
    T\kern-.1667em\lower.7ex\hbox{E}\kern-.125emX}}

\usepackage{fancyhdr}
\fancyhf{}
\pagestyle{fancyplain}
\fancyhead[R]{\thepage}

%meta
%meta

\fancyhead[L]{P. Vierkant \& M. Kindling \\ %author
LIBREAS. Library Ideas, 26 (2014). % journal, issue, volume.
\href{http://nbn-resolving.de/urn:nbn:de:kobv:11-100222687
}{urn:nbn:de:kobv:11-100222687}} % urn
\fancyhead[R]{\thepage} %page number
\fancyfoot[L] {\textit{Creative Commons BY 3.0}} %licence
\fancyfoot[R] {\textit{ISSN: 1860-7950}}

\title{\LARGE{Welche Institutionen betreiben Open-Access-Repositorien in Deutschland?}} %title %title
\author{Paul Vierkant \& Maxi Kindling} %author

\setcounter{page}{52}

\usepackage[colorlinks, linkcolor=black,citecolor=black, urlcolor=blue,
breaklinks= true]{hyperref}

\date{}
\begin{document}

\maketitle
\thispagestyle{fancyplain} 

%abstracts


\begin{abstract}
Der Artikel analysiert die Ergebnisse des „2014 Cenus on Open Access
Repositories in Germany, Austria and Switzerland" im Hinblick auf die
Institutionen, die in Deutschland Repo\-si\-torien betreiben. Der 2014
Census basiert auf einer qualitativen Inhaltsanalyse der Webseiten der
Repositorien, einer automatisierten Validierung der über das OAI-PMH
ausgelieferten Metadaten sowie einer Umfrage an die
Repositorienbetreiber im Erhebungszeitraum 6. Januar bis 13. Februar
2014. Die Forschungsdaten des 2014 Census sind auf zenodo.org verfügbar.
\end{abstract}


%body
\section*{Einführung}\label{einfuxfchrung}

Das Thema Open Access ist auf der politischen Agenda in Deutschland
angekommen.\footnote{\emph{Gesetz- und Verordnungsblatt für das Land
  Brandenburg Teil I}, 2008. ;

  \emph{Gesetz über die Hochschulen in Baden-Württemberg}, 2014.;
  \enquote{Strategie Der Bundesregierung Zum Europäischen Forschungsraum
  (EFR) - Leitlinien Und Nationale Roadmap,} 2014.
  \url{http://www.bmbf.de/pubRD/EFR-Strategie_deutsch.pdf}.} Auf Bundes-
und Länderebene wird der freie Zugang zu öffentlich finanzierter
Forschung in Strategiepapieren thematisiert. Seit der Berliner Erklärung
vor über zehn Jahren ist die Zahl der Open-Access-Repo\-si\-torien in
Deutschland beträchtlich gewachsen. Viele Universitäten, Fachhochschulen
und außeruniversitäre Forschungseinrichtungen betreiben mittlerweile
Open-Ac\-cess-Repositorien (OAR), die nicht nur den wissenschaftlichen
Publikationsoutput der Institution sichtbar machen, sondern zum Teil
durch die Integration der OAR in Forschungsinformationssysteme auch der
internen Evaluation dienen.

Das breite Spektrum der Größe, Gestaltung und der Funktionen von OAR in
Deutschland wurde erstmals im \enquote{2012 Census of Open Access
Repositories in Germany (2012 Census)} umfassend untersucht.\footnote{Vierkant,
  Paul. \enquote{2012 Census of Open Access Repositories in Germany:
  Turning Perceived Knowledge Into Sound Understanding.} \emph{D-Lib
  Magazine} 19, no. 11/12 (November 2013).
  doi:\href{http:doi.org/10.1045/november2013-vierkant}{10.1045/november2013-vierkant}.}
Der 2012 Census trug zur Verortung einzelner OAR sowie der gesamten
deutschen Repositorienlandschaft bei. Die damalige Studie wurde 2014 im
Rahmen eines Projektseminars am Institut für Bibliotheks- und
Informationswissenschaft wiederholt und in entscheidenden Punkten
erweitert.

Der \enquote{2014 Census of Open Access Repositories in Germany, Austria
and Switzerland (2014)} beinhaltet neben einem erweiterten
Kriterienkatalog für die qualitative Inhaltsanalyse der
Repo\-sitorien-Webseiten auch ein größeres Sample, das die Alpenstaaten
Österreich und Schweiz mit einbezieht. Der Census 2014 beinhaltet zudem
eine Online-Umfrage, die sich an die Betreiber von Repositorien
richtete.

Die Erweiterungen des 2014 Census ermöglichen eine umfassende Analyse
der Institutionen, die ein OAR betreiben. Der vorliegende Artikel
präsentiert die Teile des 2014 Census, die Aussagen und Rückschlüsse
über OAR-betreibende Institutionen in Deutschland zulassen. Die
nachfolgenden Ergebnisse bieten eine überblicksartige Darstellung der
deutschen Repositorienlandschaft für betreibende Institutionen von OAR,
Forschungsförderorganisationen und wissenschaftspolitische Entscheider.
Darauf aufbauend können neue Entwicklungen vorangetrieben und
strategische Entscheidungen getroffen werden, um den grünen Weg des Open
Access in Deutschland weiter auszubauen. Die Verfasser sind sehr an
einem Austausch mit diesen Akteuren zu den Hintergründen und
Herausforderungen interessiert.

\section*{Methode}\label{methode}

Der 2014 Census wurde wie oben erwähnt methodisch um eine qualitative
Befragung der Repositorienbetreiber erweitert. Außerdem wurden wichtige
Kriterien für die Analyse der Webseiten ergänzt. Dafür waren die
Umsetzbarkeit der Untersuchung der Kriterien, die Vergleichbarkeit mit
dem 2012 Census und die Relevanz für Betreiber maßgeblich. Für letztere
sind die Typisierung der OAR und ihrer Betreiber von Bedeutung. Für die
Typisierung wurden bereits vorliegende Definitionen von
Repositorientypen\footnote{DINI, Arbeitsgruppe Elektronisches
  Publizieren. \enquote{DINI-Zertifikat Für Open-Access-Repositorien Und
  -Publikationsdienste 2013.} DINI-Schriften 3. Deutsche Initiative für
  Netzwerkinformation (DINI), 2014.
  \href{http://nbn-resolving.de/urn:nbn:de:kobv:11-100217162}{urn:nbn:de:kobv:11-100217162}.}
und Betreibertypen\footnote{\enquote{Hochschulkompass - Ein Angebot Der
  Hochschulrektorenkonferenz.} Datenbank. \emph{Hochschulen in
  Deutschland}, November 19, 2014.
  \url{http://www.hochschulkompass.de/hochschulen/hochschulen-in-deutschland-die-hochschulsuche.html}.}
herangezogen.

Der Census 2014 besteht aus drei methodischen Komponenten:

\begin{itemize}
\item
  die qualitative Inhaltsanalyse der Repositorien-Webseiten;
\item
  die automatisierte Evaluation der über OAI-PMH ausgelieferten
  Metadaten eines OAR;
\item
  und einer Online-Umfrage.
\end{itemize}

Wie schon im 2012 Census wurden die Repositorien-Webseiten der OAR
intellektuell auf das Vorhandensein bestimmter Kriterien hin überprüft
und für jedes OAR separat erfasst. Je nach Kriterium wurde die
Startseite (zum Beispiel einfaches Suchfeld) oder die Landing-Page (z. B.
Nutzungsstatistiken) besucht. Der Erhebungszeitraum begann am 6. Januar
2014 und endete am 31. Januar 2014.

Mithilfe des DINI/OAN-Validators\footnote{http://oanet.cms.hu-berlin.de/validator/pages/validation\_dini.xhtml}
wurden die über OAI-PMH ausgelieferten Metadaten der einzelnen
\enquote{items} in den OAR auf ihre Konformität mit dem DINI-Zertifikat
2010 hin analysiert.\footnote{\enquote{Item} bezeichnet die laut
  \emph{Bielefeld Academic Search Engine} vorgehaltenen
  Open-Access-Volltextveröffentlichungen, die im Idealfall die
  Gesamtheit des Bestands ausmachen.\enquote{BASE FAQ - Fragen zur Suche
  und zur Trefferliste.} Datenbank. \emph{Bielefeld Academic Search
  Engine}, 2014 2004. \url{http://www.base-search.net/about/de/faq.php}.}
Beim 2012 Census wurden die Metadaten von 200 zufällig ausgewählten
items analysiert, während beim 2014 Census alle im OAR vorgehaltenen
items analysiert wurden. Das Ergebnis wurde mit einem Gesamt-Score in
einem Report ausgeliefert und dem Datensatz des OAR zugeordnet. Der
Erhebungszeitraum begann am 6. Januar 2014 und endete am 17. Januar
2014.

Der dritte und letzte Teil des Census 2014 ist eine Umfrage unter
Repositorien-Betreibern mittels eines nicht-anonymisierten
Online-Fragebogens. Mit dem Fragebogen konnten Kriterien erfasst werden,
für die die notwendigen Informationen schwierig oder gar nicht
öffentlich zugänglich waren. Als Kontaktadresse wurde die im Impressum
des jeweiligen OAR angegebene E-Mail-Adresse verwendet. Konnte auf
diesem Wege keine E-Mail-Adresse gefunden werden, wurde die Webseite des
OAR entsprechend weiter abgesucht. Die Repositorien-Betreiber wurden
während des Erhebungszeitraums mehrmals kontaktiert. Nach dem Ende des
Erhebungszeitraums wurden ausschließlich vollständig ausgefüllte
Fragebögen dem jeweiligen Datensatz des OAR zugeordnet. Die in den
Freitextfeldern bereitgestellten Informationen sowie die Angaben zu den
personellen Ressourcen eines OAR wurden erfasst, aber nicht
veröffentlicht. Der Erhebungszeitraum für die Online-Umfrage begann am
13. Januar 2014 und endete am 6. Februar 2014.

\section*{Grundgesamtheit}\label{grundgesamtheit}

Die OAR, die die Grundgesamtheit des 2014 Census bilden, wurden anhand
der nachfolgenden Definition ausgewählt:

\begin{quote}
\emph{Open-Access-Repositorien sind für den Zweck dieser Studie
institutionelle und disziplinäre Repositorien aus Deutschland,
Österreich und der Schweiz, die mehrheitlich wissenschaftliche
Open-Access-Volltextveröffentlichungen vorhalten. Die
Volltextveröffentlichungen sind durch Metadaten beschrieben, die über
eine Weboberfläche (Such- und Browsefunktionalität) recherchierbar sind.
Die Open-Access-Repositorien sind mit einer funktionierenden Base-URL
für das OAI-PMH Harvesting bei Bielefeld Academic Search Engine (BASE)
registriert. Digitale Sammlungen, Forschungsdatenrepositorien, sowie
Open-Access-Repositorien von Verlagen, University Presses und
kommerzielle Dienste sind ausgeschlossen. Erhebungsdatum: 2014-01-06}
\end{quote}

Bereits der 2012 Census hat ergeben, dass von der Gesamtheit aller
bestehenden und relevanten Repositorienverzeichnissen (OpenDOAR,
ROARMap, DINI-Liste der Repositorien, BASE) die Bielefeld Academic
Search Engine (BASE) den größten Abdeckungsgrad hat (93 \%).\footnote{Vierkant,
  Paul. \enquote{2012 Census of Open Access Repositories in Germany:
  Turning Perceived Knowledge Into Sound Understanding.} \emph{D-Lib
  Magazine} 19, no. 11/12 (November 2013).
  doi:\href{http://doi.org/10.1045/november2013-vierkant}{10.1045/november2013-vierkant}.}
Für alle OAR, die zum Stichtag der Erhebung (6. Januar 2014) in BASE
erfasst waren, stellte BASE die Datenbankeinträge zu Namen, URL,
OAI-Schnittstelle und Angaben zur Zahl der items zur Verfügung. Anhand
der oben genannten Definition wurden die OAR für den 2014 Census
ausgewählt. Das Sample umfasst 152 OAR aus Deutschland, 5 OAR aus
Österreich und 16 aus der Schweiz. Insgesamt umfasst der 2014 Census 173
OAR.

\section*{Anmerkungen zum Kriterium \enquote{Typ der
verantwortlichen
Institution}}\label{anmerkungen-zum-kriterium-typ-der-verantwortlichen-institution}

Die folgende Analyse des 2014 Census bezieht sich auf das Kriterium
\enquote{Typ der verantwortlichen Institution} eines OAR. Es wurde im
2014 Census erstmalig erhoben. Es basiert auf der Definition:

\begin{quote}
\emph{Die Feststellung des Typs der verantwortlichen Institution erfolgt
für Deutschland anhand der Typisierung der Institutionen in der Liste
der Hochschulrektorenkonferenz für Deutschland, des Bundesministerium
für Wissenschaft und Forschung für Österreich und der Rektorenkonferenz
der Schweizer Universitäten (Conférence des Recteurs des Universités
Suisses, CRUS) für die Schweiz. Alle Typen wurden in drei
Hauptkategorien zusammengefasst: University, University of Applied
Sciences, Non-university research institutions and others (Alle
Institutionen, die nicht anhand der HRK-Liste zugeordnet werden konnten,
werden als außeruniversitäre Forschungseinrichtungen und Andere
kategorisiert). Erhebungsdatum: 2014-01-06}
\end{quote}

Die Liste der Hochschulrektorenkonferenz (HRK) umfasst folgende Typen
von Institutionen in Deutschland (D), die zur besseren Vergleichbarkeit
mit den äquivalenten Institutionstypen für Österreich (A) und die
Schweiz (CH) zusammengefasst wurden:

\begin{itemize}
\item
  Universität

  \begin{itemize}
  \item
    Universitäten und Hochschulen mit Promotionsrecht (D)
  \item
    Kunst- und Musikhochschulen mit und ohne Promotionsrecht (D)
  \item
    private Universität/ Hochschule mit Promotionsrecht (A)
  \item
    kirchliche Universität/ Hochschule mit Promotionsrecht (A)
  \item
    öffentliche Universität (A)
  \item
    Universitäre Hochschule gemäß Universitätsförderungsgesetz (CH)
  \end{itemize}
\end{itemize}

\begin{itemize}
\item
  Fachhochschulen

  \begin{itemize}
  \item
    Fachhochschulen und Hochschulen ohne Promotionsrecht (D)
  \item
    Fachhochschule gemäß Fachhochschulgesetz (CH)
  \end{itemize}
\item
  Außeruniversitäre Forschungseinrichtungen und Andere (im Folgenden
  nur: außeruniversitäre Forschungseinrichtungen)
\end{itemize}

\subsection*{Ergebnisse und
Diskussion}\label{ergebnisse-und-diskussion}

Für diesen Artikel wurde das Kriterium \enquote{Typ der verantwortlichen
Institution} für die OAR in Deutschland in Relation zu anderen im 2014
Census erhobenen Kriterien analysiert, die zu ersten Interpretationen
der Ergebnisse sowie zu einer Diskussion führen. Die Auswertung der
unterschiedlichen Kriterien, sowie ihre Vergleichbarkeit (zum Beispiel zwischen
Bundesländern) bedingen den Fokus auf eine Nation und in dem
vorliegenden Fall Deutschland.\footnote{Um die Auswertungen für
  Österreich und Schweiz sowie dem gesamten D-A-CH-Raum zu ermöglichen,
  werden die Forschungsdaten aller drei Länder in einem Datensatz
  veröffentlicht, siehe Fazit.}


Im 2014 Census wurde die geographische Lage der für das OAR
verantwortlichen Institution mit folgender Definition erfasst:

\begin{quote}
\emph{Das Land/Bundesland (Deutschland), in dem die betreibende und
inhaltlich verantwortliche Institution für das Open-Access-Repositorium
laut BASE liegt. Erhebungsdatum: 2014-01-06}
\end{quote}

% latex table generated in R 3.1.0 by xtable 1.7-3 package
% Tue Dec 16 19:18:21 2014
\begin{table}[ht!]
\small
\centering
\begin{tabular}{p{4cm}p{2.5cm}p{2.5cm}p{2.5cm}p{2.5cm} lrrrr}
  \toprule
 Bundesländer & Total (in \%)  & Universität Total (in \%)  & Fachhochschule Total (in \%)  & Außer\-univer\-sitäre Forschungs\-einrichtung und Andere Total (in \%)  \\ 
\cmidrule(r){1-1} \cmidrule(r){2-2} \cmidrule(r){3-3} \cmidrule(r){4-4} \cmidrule(r){5-5}
Baden-Württemberg & 31 (20,39) & 18 (58,06) & 8 (25,81) & 5 (16,13) \\ 
  Bayern & 20 (13,16) & 13 (65,00) & 4 (20,00) & 3 (15,00) \\ 
  Berlin & 12 (7,89) & 6 (50,00) & 1 (8,33) & 5 (41,67) \\ 
  Brandenburg & 11 (7,24) & 4 (36,36) & 3 (27,27) & 4 (36,36) \\ 
  Bremen & 2 (1,32) & 1 (50,00) & 0 (0,00) & 1 (50,00) \\ 
  Hamburg & 5 (3,29) & 3 (60,00) & 0 (0,00) & 2 (40,00) \\ 
  Hessen & 7 (4,61) & 5 (71,43) & 1 (14,29) & 1 (14,29) \\ 
  Niedersachsen & 12 (7,89) & 6 (50,00) & 2 (16,67) & 4 (33,33) \\ 
  Nordrhein-Westfalen & 29 (19,08) & 14 (48,28) & 7 (24,14) & 8 (27,59) \\ 
  Rheinland-Pfalz & 4 (2,63) & 4 (100,00) & 0 (0,00) & 0 (0,00) \\ 
  Saarland & 3 (1,97) & 2 (66,67) & 0 (0,00) & 1 (33,33) \\ 
  Sachsen & 8 (5,26) & 6 (75,00) & 1 (12,50) & 1 (12,50) \\ 
  Sachsen-Anhalt & 2 (1,32) & 1 (50,00) & 0 (0,00) & 1 (50,00) \\ 
  Schleswig-Holstein & 4 (2,63) & 1 (25,00) & 0 (0,00) & 3 (75,00) \\ 
  Thüringen & 2 (1,32) & 2 (100,00) & 0 (0,00) & 0 (0,00) \\ 
\cmidrule(r){1-1} \cmidrule(r){2-2} \cmidrule(r){3-3} \cmidrule(r){4-4} \cmidrule(r){5-5}
  Gesamt BRD & 152 (100) & 86 (56,58) & 27 (17,76) & 39 (25,66) \\
   \bottomrule
\end{tabular}
\caption{Anzahl und prozentuale Verteilung von Open-Access-Repositorien nach
Bundes\-ländern}
\end{table}

In den drei bevölkerungsreichsten deutschen Bundesländern Bayern,
Nordrhein-Westfalen und Baden-Württemberg sind mit 80 OAR insgesamt
52,63 \% aller deutschen OAR angesiedelt (vgl. 54,61 \% im 2012
Census).\footnote{Vierkant, Paul. \enquote{2012 Census of Open Access
  Repositories in Germany: Turning Perceived Knowledge Into Sound
  Understanding.} \emph{D-Lib Magazine} 19, no. 11/12 (November 2013).
  doi:\href{http://doi.org/10.1045/november2013-vierkant}{10.1045/november2013-vierkant}.}

Ein Blick auf die drei unterschiedlichen Typen der verantwortlichen
Institution zeigt, dass der Großteil der untersuchten OAR mit 56,58 \%
(86 OAR) von Universitäten betrieben wird. Außeruniversitäre
Forschungseinrichtungen mit 25,66 \% (39 OAR) und Fachhochschulen mit
17,76 \% (27 OAR) betreiben deutlich weniger OAR. Nur in den beiden
Bundesländern Brandenburg und Nordrhein-Westfalen fällt der Unterschied
der prozentualen Verteilung nach Typ der verantwortlichen Institution
nicht so deutlich aus. Diese Zahlen zeigen, dass der grüne Weg des
OA-Publizierens in Deutschland hauptsächlich von Universitäten
beschritten wird.

In diesem Zusammenhang stellt sich nun die Frage, wie groß der Impact
des grünen Weges tatsächlich ist: Wie hoch ist der Anteil der
Universitäten bzw. Fachhochschulen in Deutschland, die überhaupt ein OAR
betreiben?

Zur Beantwortung dieser Frage wurde die Liste der
Hochschulrektorenkonferenz, die alle Hochschulen in Deutschland
verzeichnet, als Grundgesamtheit vorausgesetzt.\footnote{Da zum
  Zeitpunkt der Erhebung kein zentrales Register für außeruniversitäre
  Forschungseinrichtungen und Andere vorlag, konnte der Anteil dieses
  Betreibertyps nicht analysiert werden.}

% latex table generated in R 3.1.0 by xtable 1.7-3 package
% Tue Dec 16 19:27:22 2014
\begin{table}[ht]
\small
\centering
\begin{tabular}{p{4cm}p{2.5cm}p{2.5cm}p{2.5cm}p{2.5cm} lrrrr}
  \toprule
  Bundesländer & Universitäten HRK Total  & mit OAR Total (in\%)  & Fachhochschulen HRK Total  & mit OAR Total (in\%) \\ 
\cmidrule(r){1-1} \cmidrule(r){2-2} \cmidrule(r){3-3} \cmidrule(r){4-4} \cmidrule(r){5-5}
  Baden-Württemberg &  27 & 15 (55,56) &  42 & 6 (14,29) \\ 
  Bayern &  23 & 10 (43,48) &  24 & 4 (16,67) \\ 
  Berlin &  12 & 5 (41,67) &  26 & 1 (3,85) \\ 
  Brandenburg &   4 & 4 (100,00) &   6 & 3 (50,00) \\ 
  Bremen &   3 & 1 (33,33) &   3 & 0 (0,00) \\ 
  Hamburg &   7 & 3 (42,86) &   9 & 0 (0,00) \\ 
  Hessen &  13 & 5 (38,46) &  17 & 1 (5,88) \\ 
  Niedersachsen &  13 & 4 (30,77) &  14 & 2 (14,29) \\ 
  Nordrhein-Westfalen &  28 & 11 (39,29) &  38 & 3 (7,89) \\ 
  Rheinland-Pfalz &   8 & 4 (50,00) &   9 & 0 (0,00) \\ 
  Saarland &   3 & 1 (33,33) &   2 & 0 (0,00) \\ 
  Sachsen &  11 & 4 (36,36) &  12 & 1 (7,89) \\ 
  Sachsen-Anhalt &   4 & 1 (25,00) &   5 & 0 (0,00) \\ 
  Schleswig-Holstein &   5 & 1 (20,00) &   7 & 0 (0,00) \\ 
  Thüringen &   5 & 2 (40,00) &   6 & 0 (0,00) \\ 
  Mecklenburg-Vorpommern &   3 & 0 (0,00) &   4 & 0 (0,00) \\
  \cmidrule(r){1-1} \cmidrule(r){2-2} \cmidrule(r){3-3} \cmidrule(r){4-4} \cmidrule(r){5-5} 
  Gesamt BRD & 169 & 71 (42,01) & 224 & 21 (9,38) \\ 
   \bottomrule
\end{tabular}
\caption{Anteil der Hochschulen, die ein OAR betreiben pro Bundesland}
\end{table}


Tabelle 2 zeigt, dass 42,01 \% aller deutschen Universitäten und 9,38 \%
aller deutschen Fachhochschulen ein OAR betreiben. Wichtig hierbei ist
zu wissen, dass zu den Universitäten laut HRK auch Musik- und
Kunsthochschulen zählen. Baden-Württemberg, Bayern und
Nordrhein-Westfalen haben eine hohe Anzahl an Universitäten, zugleich
aber auch einen hohen Anteil an Fachhochschulen, die ein OAR betreiben.
Die drei Bundesländer entsprechen somit dem Bundesdurchschnitt.
Vorbildcharakter kann den Hochschulen im Land Brandenburg zugesprochen
werden: Alle vier Universitäten betreiben ein OAR und drei von sechs
Fachhochschulen betreiben ein OAR. Einige Bundesländer schneiden
deutlich schlechter ab: Weniger als die Hälfte der dortigen
Universitäten betreiben OAR und an den Fachhochschulen werden teilweise
gar keine OAR betrieben.

Als Ursachen dafür, dass nicht einmal die Hälfte aller deutschen
Universitäten ein OAR betreiben und den noch deutlich geringeren Anteil
der Fachhochschulen, die ein OAR betreiben, lassen sich nur Vermutungen
anstellen: Vielleicht steht dies im Zusammenhang mit mangelnder Kenntnis
oder Akzeptanz des grünen Wegs des Open-Access-Publizierens, mit
mangelnden Ressourcen oder auch einem verhältnismäßig geringen
(OA-)Publikationsaufkommen der entsprechenden Institutionen. Dieser
letzteren Forschungslücke ließe sich mit einem flächendeckenden Einsatz
von Hochschulbibliografien bzw. Forschungsinformationssystemen (FIS) und
entsprechenden gemeinsamen Datenstandards annähern.

\subsection*{Größe der Open-Access-Repositorien (Zahl der items in den
Repositorien)}\label{gruxf6uxdfe-der-open-access-repositorien-zahl-der-items-in-den-repositorien}

Neben der Anzahl der OAR pro Bundesland wurden im 2014 Census ihre
jeweilige Größe erfasst. Die Größe eines OAR meint die Zahl der im OAR
verfügbaren items (Metadatensätze). Diese Angaben wurden wie eingangs
beschrieben von BASE geliefert.

Die Analyse zeigt, dass der Anteil von OAR eines Bundeslandes partikular
deutlich vom Anteil der vorgehaltenen items des jeweiligen Bundeslandes
am Gesamtbestand des Bundesgebiets abweicht, wie die Beispiele
Baden-Württemberg und Nordrhein-Westfalen zeigen. Das bedeutet dass ein
Bundesland wie Baden-Württemberg zwar einen Anteil von 20,39 \% aller
OAR in Deutschland haben kann, dass aber der Anteil der vorgehaltenen
items von 12,30 \% der Gesamtzahl der items in OAR in Deutschland nicht
im gleichen Maße hoch ist.

Die Größenunterschiede zwischen den verschiedenen Typen der
verantwortlichen Institution fallen indes noch deutlicher aus. So
beträgt der bundesweite Anteil der in den OAR von Fachhochschulen
vorgehaltenen items nur 1,29 \%, wohingegen in den OAR der Universitäten
(54,04 \%) sowie der außeruniversitären Forschungseinrichtungen (44,67 \%)
der Großteil der items zu finden ist (siehe Tabelle 3).

% latex table generated in R 3.1.0 by xtable 1.7-3 package
% Tue Dec 16 19:27:22 2014
\begin{table}[ht]
\centering
\small
\begin{tabular}{p{3cm}p{3cm}p{3cm}p{3cm}p{3cm} lrrrr}
  \toprule
Bundesländer & Total OAR (in \% Gesamt)  & Items Total (in \% Gesamt)  & Items in OAR Uni Total (in \% Gesamt)  & Items in OAR FH Total (in \% Gesamt)  \\ 
    \cmidrule(r){1-1} \cmidrule(r){2-2} \cmidrule(r){3-3} \cmidrule(r){4-4} \cmidrule(r){5-5} 
Baden-Württemberg & 31 (20,39) & 109.326 (12,30) & 103.804 (94,95) & 1.380 (1,26) \\ 
  Bayern & 20 (13,16) & 162.013 (18,23) & 132.928 (82,05) & 104 (0,06) \\ 
  Berlin & 12 (7,89) & 73.228 (8,24) & 29.682 (40,53) & 74 (0,10) \\ 
  Brandenburg & 11 (7,24) & 19.039 (2,14) & 9.757 (51,25) & 901 (4,73) \\ 
  Bremen & 2 (1,32) & 34.803 (3,92) & 2.380 (6,84) & 0 (0,00) \\ 
  Hamburg & 5 (3,29) & 35.554 (4,00) & 6.450 (18,14) & 0 (0,00) \\ 
  Hessen & 7 (4,61) & 60.916 (6,85) & 51.020 (83,75) & 3.542 (5,81) \\ 
  Niedersachsen & 12 (7,89) & 26.391 (2,97) & 23.868 (90,44) & 357 (1,35) \\ 
  Nordrhein-Westfalen & 29 (19,08) & 227.996 (25,65) & 78.318 (34,35) & 1.665 (0,73) \\ 
  Rheinland-Pfalz & 4 (2,63) & 6.825 (0,77) & 6.825 (100,00) & 0 (0,00) \\ 
  Saarland & 3 (1,97) & 11.461 (1,29) & 8.003 (69,83) & 0 (0,00) \\ 
  Sachsen & 8 (5,26) & 17.783 (2,00) & 13.734 (77,23) & 3.473 (19,53) \\ 
  Sachsen-Anhalt & 2 (1,32) & 3.445 (0,39) & 1.081 (31,38) & 0 (0,00) \\ 
  Schleswig-Holstein & 4 (2,63) & 90.157 (10,14) & 2.620 (2,91) & 0 (0,00) \\ 
  Thüringen & 2 (1,32) & 9.800 (1,10) & 9.800 (100,00) & 0 (0,0) \\   \cmidrule(r){1-1} \cmidrule(r){2-2} \cmidrule(r){3-3} \cmidrule(r){4-4} \cmidrule(r){5-5} 
  Gesamt BRD & 152 (100,00) & 888.737 (100,00) & 480.270 (54,04) & 11.496 (1,29) \\ 
   \bottomrule
\end{tabular}
\caption{Größe von OAR nach Bundesländern}
\end{table}

Diese ungleiche Verteilung spiegelt sich ebenfalls in den
Größenunterschieden zwischen den OAR verschiedener Institutionstypen
wider. Tabelle 4 zeigt den Median der in einem deutschen OAR
vorgehaltenen items nach den Typen der verantwortlichen Institutionen.
Der Median eines OAR in Deutschland liegt bei 2.141, wobei die von
Universitäten (2.670) und außeruniversitären Forschungseinrichtungen
(2.564) betriebenen OAR darüber liegen. Mit lediglich 110 vorgehaltenen
items sind OAR, die von Fachhochschulen betrieben werden, deutlich
kleiner (siehe Tabelle 4). Auf welche Ursachen die enormen
Größenunterschiede sowie die ungleiche Verteilung auf die
unterschiedlichen Institutionstypen zurückzuführen ist, bleibt auch hier
zu überprüfen.\footnote{Vergleiche Hansche, Dorothea. \enquote{Open
  Access an Fachhochschulen am Beispiel der Fachhochschule Potsdam: Ein
  Kommunikationskonzept,} 2011, S. 34.
  \href{http://nbn-resolving.de/urn:nbn:de:kobv:525-opus-2182}{urn:nbn:de:kobv:525-opus-2182}.}

% latex table generated in R 3.1.0 by xtable 1.7-3 package
% Tue Dec 16 19:27:22 2014
\begin{table}[ht]
\centering
\small
\begin{tabular}{p{6cm} lr}
  \hline
Institutionstyp & Items (Median) \\ 
  \hline
Median eines deutschen OAR & 2141 \\ 
  Universität & 2670 \\ 
  Fachhochschulen & 110 \\ 
  Außeruniversitäre Forschungseinrichtung und Andere & 2564 \\ 
   \hline
\end{tabular}
\caption{Median der in OAR vorgehaltenen items nach Typ der
verantwortlichen Einrichtung}
\end{table}

\subsubsection*{Typen von
Open-Access-Repositorien}\label{typen-von-open-access-repositorien}

Im 2014 Census wurde neben der Typisierung der Institutionen, die ein
OAR betreiben, auch erstmals eine Typisierung der OAR selbst nach
fachbezogenen oder institutionellen Repositorien durchgeführt. Es
erfolgte keine Zuordnung nach dem Typ \enquote{Cross-institutional
repository}, weil eine eindeutige Zuordnung zu institutionell oder
cross-institutional wie in den Fällen der Repositorien von
Forschungsorganisationen (wie Max-Planck-Gesellschaft und
Fraunhofer-Gesellschaft) teilweise nicht möglich war. Für die
Typisierung wurde auf die Definitionen für fachbezogene und
institutionelle Repositoren aus dem DINI-Zertifikat für
Open-Access-Repositorien und -Publikationsdienste 2013
zurückgegriffen.\footnote{Siehe 53 DINI, Arbeitsgruppe Elektronisches
  Publizieren. \enquote{DINI-Zertifikat für Open-Access-Repositorien und
  -Publikationsdienste 2013.} DINI-Schriften 3. Deutsche Initiative für
  Netzwerkinformation (DINI), 2014, S. 52. urn:nbn:de:kobv:11-100217162.

  \enquote{Ein fachbezogenes Open-Access-Repositorium beinhaltet
  überwiegend einer bestimmten Disziplin zugehörige
  Open-Access-Volltexte. Darunter kann jegliche Art von
  wissenschaftlichen Publikationen fallen (Qualifikationsarbeiten,
  Berichte, Zweitveröffentlichungen etc.). In Fachrepositorien werden
  Publikationen von Personen zugänglich gemacht, die verschiedenen
  Institutionen angehören können.\\Ein institutionelles Repositorium
  beinhaltet überwiegend Open-Access-Volltexte einer Einrichtung.
  Darunter kann jegliche Art wissenschaftlicher Publikationen fallen
  (Qualifikationsarbeiten, Berichte, Zweitveröffentlichungen etc.).
  Darüber hinaus kann das Repositorium auch weitere Ergebnisse des
  wissenschaftlichen Alltags in digitaler Form enthalten.}}

Das Gros der OAR in Deutschland bilden die 135 institutionellen OAR
(88,82 \%) von Universitäten, Fachhochschulen und außeruniversitären
Forschungsreinrichtungen (siehe Tabelle 5). Dem gegenüber stehen 17
fachbezogene OAR (11,18 \%), die ausschließlich von außeruniversitären
Forschungseinrichtungen und Universitäten betrieben werden.
Außeruniversitäre Forschungseinrichtungen betreiben, bezogen auf ihren
Anteil an den fachbezogenen OAR, überproportional viele fachbezogene
OAR.

% latex table generated in R 3.1.0 by xtable 1.7-3 package
% Tue Dec 16 19:27:22 2014
\begin{table}[ht]
\centering
\small
\begin{tabular}{lllll}
  \toprule
Repositorientyp &  OAR Uni (in\%)   & OAR FH (in\%)  & OAR AUF (in\%)  & Total (in\%) \\ 
     \cmidrule(r){1-1} \cmidrule(r){2-2} \cmidrule(r){3-3} \cmidrule(r){4-4} \cmidrule(r){5-5}
Fachbezogenes OAR & 8 (47,06) & 0 (0,00) & 9 (52,94) & 17 (11,18) \\ 
  Institutionelles OAR & 78 (57,78) & 27 (20,00) & 30 (22,22) & 135 (88,82) \\ 
     \cmidrule(r){1-1} \cmidrule(r){2-2} \cmidrule(r){3-3} \cmidrule(r){4-4} \cmidrule(r){5-5}
  Total & 86 (56,58) & 27 (17,76) & 39 (25,66) & 152 (100,00) \\ 
   \bottomrule
\end{tabular}
\caption{Typen von Open-Access-Repositorien nach betreibenden
Institutionen}
\end{table}

\subsection*{Software von
Open-Access-Repositorien}\label{software-von-open-access-repositorien}

Eine der wichtigsten Aufgaben beim Management von OAR ist die Wahl der
Software. Im 2014 Census wurde anhand der Angaben auf den Webseiten der
OAR erfasst, welche Software verwendet wird, um das OAR zu betreiben.
Außer den Software-Lösungen DSpace, EPrints und OPUS wird keine Software
für mehr als zehn Instanzen verwendet. Alle weiteren Software-Lösungen
wie etwa MyCoRe wurden in der Kategorie \enquote{Andere}
zusammengefasst.

Mit 81 Instanzen der Software OPUS (53,29) bleibt Deutschland auch im
Jahr 2014 ein \enquote{OPUS-Land} (siehe Tabelle 6). Die Mehrheit dieser
OPUS-Instanzen wird an Universitäten und Fachhochschulen betrieben. Wie
in 2012 war es das Ziel zu erfahren ob ein OAR gehostet wird, d.h. dass
die Instanz durch einen dritten Anbieter bereitgestellt wird. Im 2014
Census wurde dieses Kriterium in der Online-Umfrage abgefragt. Durch die
geringe Rücklaufquote zu dieser Frage haben die Daten nur eine bedingte
Aussagekraft.

Die im 2012 Census gestellte Prognose einer \enquote{Konzentration und
Evolution} von Repositorien-Software wird im 2014 Census
bestätigt.\footnote{Vierkant, Paul. \enquote{2012 Census of Open Access
  Repositories in Germany: Turning Perceived Knowledge Into Sound
  Understanding.} \emph{D-Lib Magazine} 19, no. 11/12 (November 2013).
  doi:\href{http://doi.org/10.1045/november2013-vierkant}{10.1045/november2013-vierkant}.}
Zwischen 2012 und 2014 sind sechs Institutionen mit ihrer OAR-Software
migriert. Alle Betreiber wechselten von OPUS hin zu EPrints (insgesamt
fünf OAR, vier davon an der UB Heidelberg) und zu einer anderen
Software. Dieser Trend zur Nutzung von internationalen Softwarelösungen
wie EPrints wird sich vermutlich weiter fortsetzen. Diese Tendenz wird
durch Aussagen von OAR-Betreibern im Zusammenhang mit der Durchführung
des Census bestätigt, die entweder eine Migration vorbereiten oder sich
bereits in der konkreten Umsetzungsphase befinden.

% latex table generated in R 3.1.0 by xtable 1.7-3 package
% Tue Dec 16 19:27:22 2014
\begin{table}[ht]
\centering
\begin{tabular}{lllll}
  \toprule
Software & OAR Uni (in\%)   & OAR FH (in\%)  & OAR AUF (in\%)  & Total (in\%) \\  
\cmidrule(r){1-1} \cmidrule(r){2-2} \cmidrule(r){3-3} \cmidrule(r){4-4} \cmidrule(r){5-5}
DSpace & 6 (6,98) & 0 (0,00) & 5 (12,82) & 11 (7,24) \\ 
  EPrints & 18 (20,93) & 0 (0,00) & 5 (12,82) & 23 (15,13) \\ 
  OPUS & 38 (44,19) & 27 (100,00) & 16 (41,03) & 81 (53,29) \\ 
  Other & 24 (27,91) & (0,00) & 13 (33,33) & 37 (24,34) \\ 
  \cmidrule(r){1-1} \cmidrule(r){2-2} \cmidrule(r){3-3} \cmidrule(r){4-4} \cmidrule(r){5-5}
   & 86 (100,00) & 27 (100,00) & 39 (100,00) & 152 (100,00) \\ 
   \bottomrule
\end{tabular}
\caption{Software von OAR nach betreibenden Institutionen}
\end{table}

\subsection*{DINI-Zertifikat}\label{dini-zertifikat}

Als wichtiger Standard für Open-Access-Repositorien in Deutschland gilt
das DINI-Zertifikat für Dokumenten- und Publikationsservices. Es dient
Repositorien-Betreibern beim Aufbau und der Weiterentwicklung ihres OA
in technischer und organisatorischer Hinsicht.\footnote{DINI,
  Arbeitsgruppe Elektronisches Publizieren. \enquote{DINI-Zertifikat Für
  Open-Access-Repositorien Und -Publikationsdienste 2013.}
  DINI-Schriften 3. Deutsche Initiative für Netzwerkinformation (DINI),
  2014.
  \href{http://nbn-resolving.de/urn:nbn:de:kobv:11-100217162}{urn:nbn:de:kobv:11-100217162}.}
Der Census erfasst, ob das OAR laut DINI-Liste der Repositorien mit dem
Stand vom 6. Januar 2014 ein DINI-Zertifikat besitzt. Die
Zertifikatsversionen 2004/2007/2010 wurden erhoben.

Von allen 152 OAR in Deutschland haben etwa ein Drittel ein
DINI-Zertifikat, wobei der Anteil der zertifizierten OAR für die
verschiedenen Betreibertypen sehr unterschiedlich ausfällt (siehe
Tabelle 7). Während 39,53 \% aller universitären OAR ein DINI-Zertifikat
besitzen, fällt der Wert für die OAR von Fachhochschulen mit lediglich
7,41 \% sowie für außeruniversitäre Forschungseinrichtungen und Andere
mit 17,95 \% deutlich geringer aus.

% latex table generated in R 3.1.0 by xtable 1.7-3 package
% Tue Dec 16 19:27:22 2014
\begin{table}[ht]
\centering
\begin{tabular}{lllll}
  \toprule
DINI Zertifikat & OAR Uni (in\%)   & OAR FH (in\%)  & OAR AUF (in\%)  & Total (in\%) \\  
\cmidrule(r){1-1} \cmidrule(r){2-2} \cmidrule(r){3-3} \cmidrule(r){4-4} \cmidrule(r){5-5}
ohne Zertifikat & 52 (60,47) & 25 (92,59) & 32 (82,05) & 109 (71,71)  \\ 
  mit Zertifikat & 34 (39,53) & 2 (7,41) & 7 (17,95) & 43 (28,29) \\ 
  davon 2004 & 10 (11,63) & 0 (0,00) & 0 (0,00) & 10 (6,58) \\ 
  davon 2007 & 13 (15,12) & 1 (3,70\%) & 2 (5,13) & 16 (10,53) \\ 
  davon 2010 & 11 (12,79) & 1 (3,70\%) & 5 (12,82) & 17 (11,18) \\ 
  \cmidrule(r){1-1} \cmidrule(r){2-2} \cmidrule(r){3-3} \cmidrule(r){4-4} \cmidrule(r){5-5}
  Gesamt & 86 (100,00) & 27 (100,00) & 39 (100,00) & 152 (100,00) \\ 
\bottomrule
\end{tabular}
\caption{DINI-Zertifikat nach betreibenden Institutionen}
\end{table}

\subsection*{Berliner Erklärung über den offenen Zugang zu
wissenschaftlichem
Wissen}\label{berliner-erkluxe4rung-uxfcber-den-offenen-zugang-zu-wissenschaftlichem-wissen}

Institutionen, die die Berliner Erklärung (Berlin Declaration on Open
Access to Knowledge in the Sciences and the Humanities) unterzeichnet
haben, setzen damit ein wichtiges Signal für die Unterstützung des
Open-Access-Gedankens. In diesem Kriterium wurde erfasst, ob die für das
OAR verantwortliche Institution (oder bei außeruniversitären
Forschungseinrichtungen ihre \enquote{Dachorganisation}) am Stichtag 6.
Januar 2014 auf der Webseite der Unterzeichner der Berliner Erklärung
gelistet war.\footnote{\enquote{Berlin Declaration on Open Access to
  Knowledge in the Sciences and Humanities,} 14.10.2003.
  http://openaccess.mpg.de/Berliner-Erklaerung.}

Außeruniversitäre Forschungseinrichtungen und Andere stellen mit über 50
\% einen bemerkenswert hohen Anteil derjenigen Institutionen, die die
Berliner Erklärung unterzeichnet haben. Das liegt darin begründet, dass
beispielsweise mehrere Einrichtungen der Helm\-holtz-Ge\-mein\-schaft oder
Leibniz-Gemeinschaft, die beide als Dachorganisationen die Berlin
Erklärung unterzeichnet haben, OAR betreiben (siehe Tabelle 8). Im
Vergleich dazu haben 24 \% der Universitäten und 11 \% der
Fachhochschulen, die ein OAR betreiben, die Berliner Erklärung
unterzeichnet. Diese relativ geringe Zahl ist angesichts des bereits vor
über einem Jahr begangenen zehnjährigen Jubiläums der Berliner Erklärung
bemerkenswert. Der niedrige Wert von 11 \% unterstützt die Annahme, dass
das Thema Open Access an Fachhochschulen bislang noch keinen hohen
Stellenwert besitzt.

% latex table generated in R 3.1.0 by xtable 1.7-3 package
% Tue Dec 16 19:27:22 2014
\begin{table}[ht]
\centering
\begin{tabular}{p{3cm} lllll}
  \toprule
n=152 & OAR Uni (in\%)   & OAR FH (in\%)  & OAR AUF (in\%)  & Total (in\%) \\  
\cmidrule(r){1-1} \cmidrule(r){2-2} \cmidrule(r){3-3} \cmidrule(r){4-4} \cmidrule(r){5-5} \\ 
 Berlin Declaration unterzeichnet & 65 (75,58) & 24 (88,89) & 17 (43,59) & 106 (69,74) \\ 
Berlin Declaration nicht unterzeichnet haben & 21 (24,42) & 3 (11,11) & 22 (56,41) & 46 (30,26)  \\ 
\cmidrule(r){1-1} \cmidrule(r){2-2} \cmidrule(r){3-3} \cmidrule(r){4-4} \cmidrule(r){5-5}  
Total & 86 (100,00) & 27 (100,00) & 39 (100,00) & 152 (100,00) \\
\bottomrule
  \end{tabular}
  \caption{Berlin-Declaration-Unterzeichner nach betreibenden
Institutionen}
\end{table}

\subsection*{Startdatum}\label{Startdatum}

Da über den historischen Verlauf des grünen Wegs in Deutschland bis dato
wenig Überblickswissen bekannt ist, sollte im Rahmen des 2014 Census
erfasst werden, wann die an der Umfrage teilnehmenden OAR
\enquote{online gegangen} sind. Die konkrete Frage lautete:
\enquote{Seit welchem Jahr ist Ihr Repositorium über das WWW zugänglich?
Im Falle von Vorgängerversionen wird die Jahreszahl der ersten
Implementierung des Repositoriums erfasst.}

Beim Vergleich des Medians des angegeben Startdatums aller OAR geordnet
nach Betreibertypen zeigt sich, dass Universitäten früher als
Fachhochschulen und außeruniversitäre Forschungseinrichtungen und Andere
mit ihren OAR gestartet sind (siehe Tabelle 9). Diese Vorreiterrolle der
Universitäten manifestiert sich vor allem in den Gründungsjahren um die
Jahrtausendwende. Von 1998 bis 2002 gingen mit 23 der 51 mehr als die
Hälfte der universitären OAR online, die an der Umfrage teilnahmen. Dem
gegenüber stehen drei OAR von Fachhochschulen sowie außeruniversitären
Forschungseinrichtungen und Anderen, die zwischen 1997 und 2002
starteten. Dieses Ergebnis vermittelt den Eindruck, dass es
Universitäten waren, die den grünen Pfad in Deutschland zuerst
einschlugen und sowohl für Fachhochschulen als auch für
außeruniversitäre Forschungseinrichtungen ebneten. Zu ergänzen ist an
dieser Stelle, dass die Universitäten bereits Ende der 1990er Jahre mit
dem Aufbau von Dissertationsservern begonnen haben, die dann zu OAR
weiterentwickelt wurden.

% latex table generated in R 3.1.0 by xtable 1.7-3 package
% Tue Dec 16 19:27:22 2014
\begin{table}[ht]
\centering
\begin{tabular}{llll}
(n=80)(a) & Median Start OAR Uni & Median Start OAR FH & Median Start OAR AUF\\ 
  \hline
Startdatum & 2002 (51) & 2006 (9) & 2006 (21) \\ 
\end{tabular}
\caption{Median Startdatum nach betreibenden Institutionen}
\end{table}


(a) Durch einen technischen Fehler wurde bei der Erfassung des Startdatums
ein Blank-Wert zugelassen, der in dieser Auswertung nicht berücksichtigt
werden kann, weshalb sich die Grundgesamtheit von ursprünglich 81
Umfrageteilnehmern auf 80 reduziert.

\subsection*{Fazit und Ausblick}\label{fazit-und-ausblick}

Der vorliegende Artikel konzentriert sich auf die betreibenden
Institutionen von OAR in Deutschland, auch wenn dieser sich ebenso auf
Österreich und die Schweiz erstreckte. Die aufgeführten Ergebnisse des
Census 2014 spiegeln dabei nur einen kleinen Teil der Erkenntnisse
wieder, die aus den Census-Daten gewonnen werden können. Einige dieser
Ergebnisse wurden bereits im Rahmen von Konferenzen, auf Postern sowie
Grafiken präsentiert.\footnote{Alle zum 2014 Census gehörigen
  Auswertungen sind als Referenz zum Datensatz unter
  doi:\href{http://doi.org/10.5281/zenodo.10734}{10.5281/zenodo.10734}
  angegeben.}

Die Ergebnisse des 2014 Census lassen sich auch in den Gesamtkontext der
wissenschaftspolitischen Strategien zu Open Access der Bundesländer
einbetten. So lässt sich vermuten, dass es kein Zufall ist, dass
Brandenburg in seinem Landeshochschulgesetz \enquote{open access}
ausdrücklich erwähnt und gleichzeitig 70 \% aller Universitäten und
Fachhochschulen in Brandenburg ein OAR betreiben.\footnote{Siehe:
  Gesetz- und Verordnungsblatt für das Land Brandenburg Teil I, 2008.}
Brandenburg stellt damit den bundesweiten Rekord. Die starke Rolle
Baden-Württembergs mit zahlreichen OAR und dem großen Anteil an
vorgehaltenen items hält in der Gesetzgebung des Landes die Hochschulen
dazu an, den Hochschulangehörigen die Zweitveröffentlichung auf
Grundlage des Zweitveröffentlichungsrechts zu ermöglichen.\footnote{\emph{Gesetz
  über die Hochschulen in Baden-Württemberg}, 2014.}

Vor dem Hintergrund dieser aktuellen Entwicklungen fassen die
nachstehenden Lessons learned die für die Betreiber von OAR relevanten
Ergebnisse des Cenus 2014 zusammen:

Kernaussagen des 2014 Census in Bezug auf OAR betreibende Institutionen
in Deutschland sind:

\begin{enumerate}
\def\labelenumi{\arabic{enumi}.}
\item
  Universitäten sind die historischen Vorreiter im Aufbau von OAR in
  Deutschland.
\item
  Die meisten OAR in Deutschland werden von Universitäten betrieben.
\item
  Gleichzeitig betreiben weniger als die Hälfte aller Universitäten in
  Deutschland ein OAR.
\item
  Nur wenige deutsche Fachhochschulen betreiben ein OAR.
\item
  Fast alle items (Open-Access-Publikationen) in Deutschlands OAR werden
  von Universitäten sowie außeruniversitären Forschungseinrichtungen und
  Anderen vorgehalten, wobei Baden-Württemberg, Bayern und
  Nordrhein-Westfalen zusammen mehr als die Hälfte der items stellen.
\item
  OAR von Fachhochschulen sind hinsichtlich der Anzahl der items
  deutlich kleiner als OAR von Universitäten und außeruniversitären
  Forschungseinrichtungen und Anderen.
\item
  Alle fachbezogenen OAR in Deutschland werden an Universitäten sowie
  außeruniversitären Forschungseinrichtungen und Anderen betrieben.
\item
  Mehr als die Hälfte aller OAR in Deutschland werden mit der
  OAR-Software OPUS betrieben. Bei den zwischen 2012 und 2014 gezählten
  Software-Migrationen wechselten alle OAR von OPUS zu einer anderen
  Software-Lösung.
\item
  Ein Drittel aller deutschen OAR besitzt ein DINI-Zertifikat.
\item
  Nach über zehn Jahren Berliner Erklärung haben ein Drittel aller
  deutschen OAR-Betreiber diese auch unterzeichnet.
\end{enumerate}

In Deutschland spielen Open-Access-Repositorien eine wichtige Rolle bei
der Umsetzung und Förderung von Open Access. Die Betreiber von OAR
müssen beständig auf wissenschaftspolitische Vorgaben, Anforderungen von
Wissenschaftlerinnen und Wissenschaftlern sowie technische Neuerungen
reagieren. Auf Grundlage des 2014 Census können Repositorienbetreiber
Strategien zur Weiterentwicklung ihrer Dienste neu ausrichten und an die
Bedürfnisse ihrer Institution anpassen. In diesem Sinne stellen die
Daten des 2014 Census einen wertvollen Beitrag für die interne
Evaluation und Verortung auch auf Landes- und Bundesebene dar, der den
Blick auf den grünen Weg freimacht, der noch vor uns liegt.

\emph{Forschungsdaten}

Forschungsdaten zum ``2014 Census on Open Access Repositories in
Germany, Austria and Switzerland'' unter:
\href{http://doi.org/10.5281/zenodo.10734}{10.5281/zenodo.10734}

\emph{Danksagung}

Unser herzlicher Dank für die gemeinsame Konzeptionierung und
Durchführung des „2014 Census on Open Access Repositories in Germany,
Austria and Switzerland" gilt Dennis Zielke (Computer- und Medienservice
der HU Berlin), Marleen Burger, Anne Lepke, Thomas Maluck, Jessika
Rücknagel, Stephanie van de Sandt und Lisa Theileis (Studierende am
Institut für Bibliotheks- und Informationswissenschaft der HU Berlin)
sowie Friedrich Summann (Universitätsbibliothek Bielefeld) für die
Bereitstellung der BASE-Daten.

%autor

\end{document}